% ============================================
% PARTIE 4 : CONCLUSION
% ============================================
\newpage
\part{Conclusion et Perspectives}

\section{Synthèse des Résultats}

\subsection{Objectifs Atteints}

Ce projet a permis de démontrer la faisabilité d'un système de tri automatisé de CSR basé sur l'intelligence artificielle. Les principaux objectifs ont été atteints :

\begin{table}[H]
\centering
\begin{tabular}{|l|c|c|c|}
\hline
\textbf{Objectif} & \textbf{Cible} & \textbf{Résultat} & \textbf{Statut} \\
\hline
Précision de tri & $\geq$ 95\% & 97.2\% & \cmark \\
Détection PVC & $\geq$ 90\% & 95\% & \cmark \\
Cadence & $\geq$ 80 obj/min & 100 obj/min & \cmark \\
Taux de chlore & $<$ 1\% & 0.3\% & \cmark \\
ROI & $<$ 24 mois & 18 mois & \cmark \\
Simulation 3D & Fonctionnelle & Opérationnelle & \cmark \\
\hline
\end{tabular}
\caption{Bilan des objectifs du projet}
\end{table}

\subsection{Contributions Principales}

\begin{enumerate}
    \item \textbf{Analyse de marché approfondie} : État des lieux complet du secteur des CSR en France et en Europe, identification des besoins non satisfaits et des opportunités.
    
    \item \textbf{Architecture innovante} : Conception d'un système combinant imagerie multispectrale (RGB + NIR + SWIR), intelligence artificielle (YOLOv8) et contrôle qualité intégré.
    
    \item \textbf{Simulation 3D interactive} : Développement d'une plateforme de démonstration permettant de visualiser le fonctionnement du système en temps réel.
    
    \item \textbf{Évaluation comparative} : Analyse rigoureuse des performances par rapport aux solutions existantes, démontrant les avantages de NATRAN.
\end{enumerate}

\subsection{Apports Scientifiques et Techniques}

\begin{figure}[H]
\centering
\begin{tikzpicture}[scale=0.8]
    % Cercles concentriques
    \fill[natranblue!10] (0,0) circle (4);
    \fill[natranblue!20] (0,0) circle (3);
    \fill[natranblue!30] (0,0) circle (2);
    \fill[natranblue!50] (0,0) circle (1);
    
    % Labels
    \node[font=\small\bfseries, white] at (0,0) {IA};
    \node[font=\tiny] at (0,1.5) {Vision};
    \node[font=\tiny] at (0,2.5) {Robotique};
    \node[font=\tiny] at (0,3.5) {Industrie};
    
    % Flèches
    \draw[->, thick, natrangreen] (4.5,0) -- (5.5,0);
    \node[font=\small, align=left] at (7.5,0) {Impact\\croissant};
\end{tikzpicture}
\caption{Niveaux d'impact du projet NATRAN}
\end{figure}

\section{Limites et Axes d'Amélioration}

\subsection{Limites de l'Étude}

\begin{tcolorbox}[colback=natranorange!10, colframe=natranorange, title=Limites Identifiées]
\begin{enumerate}
    \item \textbf{Simulation vs Réalité} : Les performances simulées sont basées sur des études académiques et non sur des tests en conditions réelles avec de vrais CSR.
    
    \item \textbf{Variabilité des flux} : La simulation utilise une composition fixe (40\% conformes, 50\% non-conformes, 10\% incertains) qui peut différer des flux réels.
    
    \item \textbf{Conditions environnementales} : Les effets de la poussière, de l'humidité et de l'éclairage variable n'ont pas été modélisés.
    
    \item \textbf{Coûts estimatifs} : Les coûts présentés sont des estimations basées sur des données de marché et peuvent varier selon les fournisseurs.
\end{enumerate}
\end{tcolorbox}

\subsection{Axes d'Amélioration}

\begin{table}[H]
\centering
\begin{tabular}{|l|p{5cm}|c|}
\hline
\textbf{Axe} & \textbf{Description} & \textbf{Priorité} \\
\hline
Entraînement réel & Collecter des données CSR réelles pour entraîner YOLOv8 & Haute \\
\hline
Tests terrain & Valider les performances sur site industriel & Haute \\
\hline
Optimisation vitesse & Réduire la latence de détection à < 30ms & Moyenne \\
\hline
Détection thermique & Ajouter une caméra thermique pour les matériaux noirs & Moyenne \\
\hline
Interface opérateur & Développer un dashboard de supervision avancé & Basse \\
\hline
\end{tabular}
\caption{Axes d'amélioration prioritaires}
\end{table}

\section{Perspectives}

\subsection{Court Terme (6-12 mois)}

\begin{itemize}
    \item \textbf{Validation industrielle} : Partenariat avec un exploitant de CSR pour tests en conditions réelles
    \item \textbf{Entraînement du modèle} : Constitution d'un dataset de CSR réels et fine-tuning de YOLOv8
    \item \textbf{Certification} : Obtention des certifications CE et conformité réglementaire
\end{itemize}

\subsection{Moyen Terme (1-2 ans)}

\begin{itemize}
    \item \textbf{Commercialisation} : Lancement commercial auprès des PME du secteur
    \item \textbf{Évolutions techniques} : Intégration de nouvelles modalités de détection
    \item \textbf{Expansion géographique} : Déploiement en Europe (Allemagne, Italie, Espagne)
\end{itemize}

\subsection{Long Terme (3-5 ans)}

\begin{itemize}
    \item \textbf{Plateforme SaaS} : Offre de tri en tant que service avec maintenance prédictive
    \item \textbf{Diversification} : Extension à d'autres flux de déchets (DEEE, plastiques, textiles)
    \item \textbf{Intelligence collective} : Apprentissage fédéré entre installations pour amélioration continue
\end{itemize}

\begin{figure}[H]
\centering
\begin{tikzpicture}[scale=0.8]
    % Timeline
    \draw[->, very thick, natranblue] (0,0) -- (14,0);
    
    % Années
    \foreach \x/\year in {0/2025, 3/2026, 6/2027, 9/2028, 12/2029} {
        \draw[thick] (\x,-0.2) -- (\x,0.2);
        \node[font=\small] at (\x,-0.5) {\year};
    }
    
    % Jalons
    \fill[natrangreen] (1,0) circle (0.2);
    \node[font=\tiny, align=center] at (1,1) {Simulation\\terminée};
    
    \fill[natranorange] (3,0) circle (0.2);
    \node[font=\tiny, align=center] at (3,1) {Tests\\industriels};
    
    \fill[natranblue] (5,0) circle (0.2);
    \node[font=\tiny, align=center] at (5,1) {Lancement\\commercial};
    
    \fill[natranblue] (8,0) circle (0.2);
    \node[font=\tiny, align=center] at (8,1) {Expansion\\Europe};
    
    \fill[natranblue] (11,0) circle (0.2);
    \node[font=\tiny, align=center] at (11,1) {Plateforme\\SaaS};
\end{tikzpicture}
\caption{Roadmap du projet NATRAN}
\end{figure}

\section{Conclusion Générale}

Le projet NATRAN démontre le potentiel de l'intelligence artificielle pour révolutionner le tri des Combustibles Solides de Récupération. Face aux limites du tri manuel (coût, précision, sécurité), notre solution propose une alternative performante et économiquement viable.

\vspace{0.5cm}

Les résultats de simulation sont encourageants :
\begin{itemize}
    \item \textbf{Précision de 97.2\%} contre 85\% pour le tri manuel
    \item \textbf{Détection du PVC à 95\%} grâce à l'imagerie SWIR
    \item \textbf{ROI de 18 mois} avec des économies de 395 k€ sur 5 ans
    \item \textbf{Contrôle qualité intégré} réduisant les erreurs de 80\%
\end{itemize}

\vspace{0.5cm}

\begin{tcolorbox}[colback=natranblue!5, colframe=natranblue]
\centering
\textbf{NATRAN représente une avancée significative vers l'automatisation intelligente du tri des déchets industriels, contribuant à l'économie circulaire et à la transition énergétique.}
\end{tcolorbox}

\vspace{1cm}

% ============================================
% DISCLAIMER
% ============================================
\section*{Disclaimer}
\addcontentsline{toc}{section}{Disclaimer}

\begin{tcolorbox}[colback=natranred!5, colframe=natranred, title=\faExclamationTriangle\ Note Importante sur les Données]

\textbf{Concernant les chiffres présentés dans ce rapport :}

\vspace{0.3cm}

\textbf{1. Données du tri manuel (réelles) :}
\begin{itemize}
    \item Les performances du tri manuel (précision 80-85\%, cadence 20-30 obj/min, coûts) sont issues d'études industrielles et académiques publiées.
    \item Sources : ADEME, FEDEREC, études sectorielles européennes.
\end{itemize}

\vspace{0.3cm}

\textbf{2. Données de la simulation NATRAN :}
\begin{itemize}
    \item Les performances simulées (précision 97.2\%, rappel 95.6\%) sont basées sur les performances documentées de YOLOv8 sur des tâches de détection d'objets similaires.
    \item \textbf{Ces chiffres ne proviennent PAS de tests sur de vrais CSR}, car nous n'avons pas eu accès à un flux réel de Combustibles Solides de Récupération.
    \item Les probabilités de faux positifs (2.8\%) et faux négatifs (4.4\%) sont extrapolées à partir d'études sur YOLOv8 appliqué à la détection de déchets.
\end{itemize}

\vspace{0.3cm}

\textbf{3. Implications :}
\begin{itemize}
    \item Les résultats présentés sont des \textbf{projections théoriques} basées sur l'état de l'art.
    \item Une validation en conditions réelles serait nécessaire avant tout déploiement industriel.
    \item Les performances réelles pourraient varier en fonction de la qualité des images, de la variabilité des matériaux et des conditions environnementales.
\end{itemize}

\end{tcolorbox}

% ============================================
% RÉFÉRENCES
% ============================================
\newpage
\section*{Références}
\addcontentsline{toc}{section}{Références}

\begin{enumerate}
    \item ADEME (2023). \textit{État des lieux de la filière CSR en France}. Rapport annuel.
    
    \item FEDEREC (2023). \textit{Rapport d'activité de la fédération des entreprises du recyclage}.
    
    \item Ultralytics (2023). \textit{YOLOv8: State-of-the-art object detection}. Documentation technique.
    
    \item European Commission (2022). \textit{Solid Recovered Fuels - Specifications and classes}. EN ISO 21640.
    
    \item Wang, C. et al. (2023). \textit{Deep learning for waste sorting: A comprehensive review}. Waste Management, 150, 234-251.
    
    \item TOMRA Systems (2022). \textit{Sensor-based sorting technologies for waste management}. White paper.
    
    \item Zheng, Y. et al. (2022). \textit{NIR spectroscopy for plastic identification in waste streams}. Resources, Conservation and Recycling, 178, 106023.
    
    \item Ministère de la Transition Écologique (2023). \textit{Feuille de route économie circulaire}.
    
    \item Babylon.js Documentation (2024). \textit{3D engine for web applications}. https://doc.babylonjs.com
    
    \item React Documentation (2024). \textit{A JavaScript library for building user interfaces}. https://react.dev
\end{enumerate}

% ============================================
% ANNEXES
% ============================================
\newpage
\section*{Annexes}
\addcontentsline{toc}{section}{Annexes}

\subsection*{A. Liens du Projet}

\begin{itemize}
    \item \textbf{GitHub} : \url{https://github.com/selyan-mhli/projet_Natran}
    \item \textbf{Documentation technique} : Disponible dans le repository
\end{itemize}

\subsection*{B. Structure du Code Source}

\begin{verbatim}
projet_Natran/
├── src/
│   ├── components/
│   │   ├── FinalSimulation.tsx   # Simulation 3D Babylon.js
│   │   ├── Dashboard.tsx         # Monitoring temps réel
│   │   ├── Architecture.tsx      # Schéma du système
│   │   ├── Impact.tsx            # Page impact environnemental
│   │   └── Header.tsx            # Navigation
│   ├── context/
│   │   └── SimulationContext.tsx # État global (stats)
│   └── App.tsx                   # Point d'entrée
├── index.html
├── package.json
├── vite.config.ts
└── tailwind.config.js
\end{verbatim}

\subsection*{C. Technologies Utilisées}

\begin{table}[H]
\centering
\begin{tabular}{|l|l|l|}
\hline
\textbf{Catégorie} & \textbf{Technologie} & \textbf{Version} \\
\hline
Framework & React & 18.x \\
Langage & TypeScript & 5.x \\
3D Engine & Babylon.js & 6.x \\
Styling & TailwindCSS & 3.x \\
Build & Vite & 5.x \\
Graphiques & Recharts & 2.x \\
Icônes & Lucide React & 0.x \\
\hline
\end{tabular}
\caption{Stack technique du projet}
\end{table}

\subsection*{D. Glossaire}

\begin{description}
    \item[CSR] Combustible Solide de Récupération
    \item[PCI] Pouvoir Calorifique Inférieur (MJ/kg)
    \item[PVC] Polychlorure de Vinyle
    \item[NIR] Near Infrared (Proche Infrarouge)
    \item[SWIR] Short-Wave Infrared (Infrarouge à Ondes Courtes)
    \item[YOLOv8] You Only Look Once version 8 (modèle de détection)
    \item[mAP] mean Average Precision
    \item[FP] Faux Positif
    \item[FN] Faux Négatif
    \item[TP] Vrai Positif (True Positive)
    \item[TN] Vrai Négatif (True Negative)
    \item[ROI] Return On Investment
    \item[TCO] Total Cost of Ownership
\end{description}

\end{document}
