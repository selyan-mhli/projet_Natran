\documentclass[11pt,a4paper]{article}

% ============================================
% PACKAGES
% ============================================
\usepackage[utf8]{inputenc}
\usepackage[T1]{fontenc}
\usepackage[french]{babel}
\usepackage{geometry}
\usepackage{graphicx}
\usepackage{xcolor}
\usepackage{tikz}
\usetikzlibrary{shapes.geometric, arrows, positioning, calc, patterns, decorations.pathreplacing, backgrounds}
\usepackage{pgfplots}
\pgfplotsset{compat=1.18}
\usepackage{fancyhdr}
\usepackage{titlesec}
\usepackage{enumitem}
\usepackage{booktabs}
\usepackage{tabularx}
\usepackage{longtable}
\usepackage{multirow}
\usepackage{amsmath}
\usepackage{hyperref}
\usepackage{float}
\usepackage{caption}
\usepackage{subcaption}
\usepackage{tcolorbox}
\usepackage{fontawesome5}
\usepackage{pifont}
\usepackage{setspace}

% ============================================
% CONFIGURATION
% ============================================
\geometry{margin=2.5cm, top=3cm, bottom=3cm}
\setstretch{1.15}

\definecolor{natranblue}{RGB}{30, 58, 95}
\definecolor{natrangreen}{RGB}{34, 197, 94}
\definecolor{natranorange}{RGB}{249, 115, 22}
\definecolor{natrangray}{RGB}{100, 116, 139}
\definecolor{natranred}{RGB}{239, 68, 68}
\definecolor{natranyellow}{RGB}{234, 179, 8}
\definecolor{lightgray}{RGB}{243, 244, 246}

\hypersetup{colorlinks=true, linkcolor=natranblue, urlcolor=natranblue}

\pagestyle{fancy}
\fancyhf{}
\fancyhead[L]{\textcolor{natrangray}{\small NATRAN - Tri Intelligent de CSR}}
\fancyhead[R]{\textcolor{natrangray}{\small \thepage}}
\fancyfoot[C]{\textcolor{natrangray}{\small UTT - 2025}}
\renewcommand{\headrulewidth}{0.4pt}
\renewcommand{\footrulewidth}{0.4pt}

\titleformat{\section}{\Large\bfseries\color{natranblue}}{\thesection.}{0.5em}{}[\titlerule]
\titleformat{\subsection}{\large\bfseries\color{natranblue}}{\thesubsection}{0.5em}{}
\titleformat{\subsubsection}{\normalsize\bfseries\color{natranblue}}{\thesubsubsection}{0.5em}{}

\newcommand{\cmark}{\textcolor{natrangreen}{\ding{51}}}
\newcommand{\xmark}{\textcolor{natranred}{\ding{55}}}

\begin{document}

% PAGE DE GARDE
\begin{titlepage}
\centering
\vspace*{1cm}
\begin{tikzpicture}
    \fill[natranblue] (0,0) rectangle (4,1.5);
    \node[white, font=\bfseries\Large] at (2,0.75) {UTT};
\end{tikzpicture}
\vspace{1cm}

{\Huge\bfseries\color{natranblue} NATRAN\par}
\vspace{0.3cm}
{\LARGE\color{natrangray} Système Intelligent de Tri de CSR\par}
{\large\color{natrangray} par Intelligence Artificielle\par}
\vspace{2cm}

\begin{tikzpicture}[scale=0.8]
    \fill[natranblue!5] (-5,-2.5) rectangle (5,2.5);
    \draw[natranblue, thick] (-5,-2.5) rectangle (5,2.5);
    \fill[natrangray!30] (-4,-1.5) rectangle (4,-1);
    \foreach \x in {-3.5,-3,...,3.5} {\draw[natrangray] (\x,-1.5) -- (\x,-1);}
    \fill[natrangreen] (-2,-0.8) rectangle (-1.5,-0.3);
    \fill[natranred] (0,-0.8) rectangle (0.5,-0.3);
    \fill[natranblue] (2,-0.8) rectangle (2.5,-0.3);
    \fill[natrangray] (-0.2,1) rectangle (0.2,1.6);
    \draw[natranyellow, dashed] (-0.2,1) -- (-1.5,-0.3);
    \draw[natranyellow, dashed] (0.2,1) -- (1.5,-0.3);
    \node[font=\small, natranblue] at (0,-2) {Simulation 3D Interactive};
\end{tikzpicture}
\vspace{2cm}

{\Large\bfseries Rapport de Projet\par}
\vspace{0.3cm}
{\large Décembre 2025\par}
\vfill
\begin{tabular}{ll}
\textbf{Auteurs :} & Moss'Ab Mirande-Ney, Selyan Mahli \\
\textbf{Formation :} & Université de Technologie de Troyes \\
\end{tabular}
\end{titlepage}

% RÉSUMÉ
\newpage
\section*{Résumé}
\addcontentsline{toc}{section}{Résumé}

\begin{tcolorbox}[colback=natranblue!5, colframe=natranblue]
Ce rapport présente \textbf{NATRAN}, un système de tri automatisé des CSR basé sur l'IA. Face aux limites du tri manuel, nous proposons une solution combinant vision par ordinateur (YOLOv8), imagerie multispectrale et robotique. Les résultats démontrent une précision de \textbf{97.2\%} contre 85\% pour le tri manuel.
\end{tcolorbox}

\textbf{Mots-clés :} CSR, Intelligence Artificielle, YOLOv8, Tri automatisé, Pyro-gazéification

% TABLE DES MATIÈRES
\newpage
\tableofcontents
\newpage
\listoffigures
\listoftables

% INTRODUCTION
\newpage
\section*{Introduction Générale}
\addcontentsline{toc}{section}{Introduction Générale}

Les \textbf{Combustibles Solides de Récupération (CSR)} représentent une alternative aux combustibles fossiles. Cependant, leur qualité dépend de l'efficacité du tri.

\begin{tcolorbox}[colback=natranorange!10, colframe=natranorange, title=Problématique]
\textbf{Comment automatiser le tri des CSR pour garantir une qualité constante tout en réduisant les coûts et l'exposition aux matériaux dangereux ?}
\end{tcolorbox}

Ce rapport est structuré en quatre parties :
\begin{enumerate}
    \item \textbf{Analyse du marché} : État des lieux du secteur des CSR
    \item \textbf{Notre solution} : Présentation détaillée de NATRAN
    \item \textbf{Comparatif} : Analyse comparative avec les solutions existantes
    \item \textbf{Conclusion} : Bilan et perspectives
\end{enumerate}

% PARTIE 1
\newpage
\part{Analyse du Marché des CSR}

\section{Le Marché des Combustibles Solides de Récupération}

\subsection{Définition et Cadre Réglementaire}

Les CSR sont des combustibles préparés à partir de déchets non dangereux, définis par l'arrêté du 23 mai 2016.

\begin{figure}[H]
\centering
\begin{tikzpicture}[scale=0.7]
    \fill[natrangreen!30] (-3,0) -- (0,3) -- (3,0) -- cycle;
    \draw[natrangreen, thick] (-3,0) -- (0,3) -- (3,0) -- cycle;
    \draw[natrangreen] (-2.5,0.5) -- (2.5,0.5);
    \draw[natrangreen] (-2,1) -- (2,1);
    \draw[natrangreen] (-1.5,1.5) -- (1.5,1.5);
    \draw[natrangreen] (-1,2) -- (1,2);
    \node[font=\tiny] at (0,0.25) {Élimination};
    \node[font=\tiny, font=\bfseries] at (0,0.75) {CSR};
    \node[font=\tiny] at (0,1.25) {Recyclage};
    \node[font=\tiny] at (0,1.75) {Réemploi};
    \node[font=\tiny] at (0,2.5) {Prévention};
    \draw[natranorange, very thick] (-2.7,0.4) rectangle (2.7,0.9);
\end{tikzpicture}
\caption{Place des CSR dans la hiérarchie des déchets}
\end{figure}

\begin{table}[H]
\centering
\begin{tabular}{|l|c|c|c|}
\hline
\textbf{Paramètre} & \textbf{Classe 1} & \textbf{Classe 2} & \textbf{Classe 3} \\
\hline
PCI (MJ/kg) & $\geq$ 25 & $\geq$ 20 & $\geq$ 15 \\
Chlore (\%) & $\leq$ 0.2 & $\leq$ 0.6 & $\leq$ 1.0 \\
Mercure (mg/MJ) & $\leq$ 0.02 & $\leq$ 0.03 & $\leq$ 0.08 \\
\hline
\end{tabular}
\caption{Classification des CSR (NF EN ISO 21640)}
\end{table}

\subsection{Le Marché Européen}

\begin{figure}[H]
\centering
\begin{tikzpicture}
\begin{axis}[
    width=11cm, height=6cm, ybar, bar width=12pt,
    xlabel={Année}, ylabel={Production (Mt)},
    ymin=0, ymax=18,
    xtick={2018,2019,2020,2021,2022,2023,2024},
    nodes near coords, nodes near coords align={vertical},
    title={\textbf{Production de CSR en Europe}},
]
\addplot[fill=natranblue!70] coordinates {
    (2018,8.2) (2019,9.1) (2020,8.5) (2021,10.3) (2022,11.8) (2023,13.2) (2024,14.5)
};
\end{axis}
\end{tikzpicture}
\caption{Évolution de la production de CSR en Europe}
\end{figure}

\begin{figure}[H]
\centering
\begin{tikzpicture}
\begin{axis}[
    width=9cm, height=6cm, xbar, bar width=10pt,
    xlabel={Production (Mt)},
    symbolic y coords={Autres, Espagne, France, Italie, UK, Allemagne},
    ytick=data, nodes near coords,
    title={\textbf{Production par pays (2023)}},
]
\addplot[fill=natranblue!70] coordinates {
    (4.2,Allemagne) (2.8,UK) (2.1,Italie) (1.9,France) (1.1,Espagne) (1.1,Autres)
};
\end{axis}
\end{tikzpicture}
\caption{Répartition par pays}
\end{figure}

\subsection{Le Marché Français}

\begin{tcolorbox}[colback=lightgray, colframe=natrangray, title=Chiffres Clés France 2023]
\begin{itemize}
    \item \textbf{Production} : 1.9 Mt/an
    \item \textbf{Installations} : 47 sites
    \item \textbf{Emplois} : 2,500 personnes
    \item \textbf{CA} : 450 M€
\end{itemize}
\end{tcolorbox}

\begin{table}[H]
\centering
\begin{tabular}{|l|c|}
\hline
\textbf{Entreprise} & \textbf{Part de marché} \\
\hline
Veolia & 28\% \\
Suez & 22\% \\
Paprec & 15\% \\
Séché Environnement & 12\% \\
Autres & 23\% \\
\hline
\end{tabular}
\caption{Principaux acteurs français}
\end{table}

\section{Technologies de Tri Existantes}

\subsection{Le Tri Manuel}

\begin{figure}[H]
\centering
\begin{tikzpicture}[scale=0.7]
    \fill[natrangray!30] (-5,-0.3) rectangle (5,0.2);
    \foreach \x in {-4.5,-4,...,4.5} {\draw[natrangray] (\x,-0.3) -- (\x,0.2);}
    \foreach \x/\col in {-3/natrangreen, -1/natranred, 1/natranorange, 3/natranblue} {
        \fill[\col] (\x,0.3) rectangle (\x+0.3,0.6);
    }
    \foreach \x in {-2, 2} {
        \draw[thick] (\x,1.3) circle (0.2);
        \draw[thick] (\x,1.1) -- (\x,0.6);
        \draw[thick] (\x,0.9) -- (\x-0.2,0.5);
        \draw[thick] (\x,0.9) -- (\x+0.2,0.5);
    }
    \fill[natrangreen!30] (-4.5,-1.3) rectangle (-3.5,-0.8);
    \fill[natranred!30] (3.5,-1.3) rectangle (4.5,-0.8);
    \node[font=\tiny] at (-4,-1.05) {Accept};
    \node[font=\tiny] at (4,-1.05) {Reject};
\end{tikzpicture}
\caption{Principe du tri manuel}
\end{figure}

\begin{table}[H]
\centering
\begin{tabular}{|l|c|l|}
\hline
\textbf{Indicateur} & \textbf{Valeur} & \textbf{Commentaire} \\
\hline
Précision & 80-85\% & Variable selon fatigue \\
Cadence & 20-30 obj/min & Par opérateur \\
Coût horaire & 25-35 €/h & Charges incluses \\
Erreur chlore & 8-12\% & Risque contamination \\
\hline
\end{tabular}
\caption{Performances du tri manuel}
\end{table}

\begin{tcolorbox}[colback=natranred!10, colframe=natranred, title=Limites du Tri Manuel]
\begin{enumerate}
    \item Fatigue et variabilité des performances
    \item Exposition aux matériaux dangereux
    \item Coût élevé (40-60\% des coûts opérationnels)
    \item Difficulté de recrutement
    \item Incapacité à détecter certains contaminants (PVC vs PET)
\end{enumerate}
\end{tcolorbox}

\subsection{Technologies Automatisées Existantes}

\begin{table}[H]
\centering
\begin{tabular}{|l|c|c|c|}
\hline
\textbf{Technologie} & \textbf{Investissement} & \textbf{Précision} & \textbf{Détection PVC} \\
\hline
NIR & 300-500 k€ & 90\% & Limitée \\
Rayons X & 500-800 k€ & 92\% & \cmark \\
IA (existant) & 200-400 k€ & 88\% & Variable \\
\hline
\end{tabular}
\caption{Comparaison des technologies existantes}
\end{table}

\section{Synthèse et Opportunités}

\begin{figure}[H]
\centering
\begin{tikzpicture}[scale=0.8]
    \draw[thick] (0,0) -- (8,0) -- (8,6) -- (0,6) -- cycle;
    \draw[thick] (4,0) -- (4,6);
    \draw[thick] (0,3) -- (8,3);
    \fill[natrangreen!20] (0,3) rectangle (4,6);
    \fill[natranblue!20] (4,3) rectangle (8,6);
    \fill[natranorange!20] (0,0) rectangle (4,3);
    \fill[natranred!20] (4,0) rectangle (8,3);
    \node[font=\bfseries\small] at (2,5.5) {FORCES};
    \node[font=\bfseries\small] at (6,5.5) {FAIBLESSES};
    \node[font=\bfseries\small] at (2,2.5) {OPPORTUNITÉS};
    \node[font=\bfseries\small] at (6,2.5) {MENACES};
    \node[font=\tiny, align=left] at (2,4.5) {Marché croissant\\Demande forte\\Réglementation};
    \node[font=\tiny, align=left] at (6,4.5) {Coûts élevés\\Complexité\\ROI incertain};
    \node[font=\tiny, align=left] at (2,1.5) {IA et deep learning\\Pénurie main d'œuvre\\Économie circulaire};
    \node[font=\tiny, align=left] at (6,1.5) {Concurrence\\Évolution réglementaire\\Résistance changement};
\end{tikzpicture}
\caption{Analyse SWOT du marché}
\end{figure}

% ============================================
% PARTIE 2 : NOTRE SOLUTION
% ============================================
\newpage
\part{Notre Solution : NATRAN}

\section{Présentation Générale}

\subsection{Vision et Objectifs}

\textbf{NATRAN} (Natural Transformation) est un système de tri intelligent combinant :
\begin{itemize}
    \item Vision par ordinateur avec caméras multispectrales
    \item Intelligence artificielle (YOLOv8)
    \item Robotique automatisée
    \item Contrôle qualité intégré
\end{itemize}

\begin{tcolorbox}[colback=natranblue!5, colframe=natranblue, title=Notre Vision]
\textbf{Démocratiser l'accès au tri automatisé de haute précision pour tous les acteurs de la filière CSR.}
\end{tcolorbox}

\subsection{Architecture du Système}

\begin{figure}[H]
\centering
\begin{tikzpicture}[scale=0.65, node distance=0.8cm,
    box/.style={rectangle, draw=natranblue, thick, fill=natranblue!10, minimum width=2.2cm, minimum height=0.9cm, align=center, font=\tiny}]
    
    \node[box, fill=natrangreen!20] (cam1) at (-3,3) {Caméra RGB};
    \node[box, fill=natrangreen!20] (cam2) at (0,3) {Caméra NIR};
    \node[box, fill=natrangreen!20] (cam3) at (3,3) {Caméra SWIR};
    \node[box, fill=natranorange!20] (fusion) at (0,1.5) {Fusion Multimodale};
    \node[box, fill=natranorange!20] (yolo) at (0,0) {YOLOv8 Détection};
    \node[box, fill=natranblue!20] (decision) at (0,-1.5) {Moteur Décision};
    \node[box, fill=natranred!20] (robot1) at (-2.5,-3) {Bras Accept};
    \node[box, fill=natranred!20] (robot2) at (2.5,-3) {Bras Reject};
    \node[box, fill=natranyellow!20] (qc) at (0,-4.5) {Contrôle Qualité};
    
    \draw[->, thick, natranblue] (cam1) -- (fusion);
    \draw[->, thick, natranblue] (cam2) -- (fusion);
    \draw[->, thick, natranblue] (cam3) -- (fusion);
    \draw[->, thick, natranblue] (fusion) -- (yolo);
    \draw[->, thick, natranblue] (yolo) -- (decision);
    \draw[->, thick, natranblue] (decision) -- (robot1);
    \draw[->, thick, natranblue] (decision) -- (robot2);
    \draw[->, thick, natranblue] (robot1) |- (qc);
    \draw[->, thick, natranblue] (robot2) |- (qc);
\end{tikzpicture}
\caption{Architecture en couches du système NATRAN}
\end{figure}

\section{Composants Techniques}

\subsection{Système de Vision Multispectrale}

\begin{table}[H]
\centering
\begin{tabular}{|l|c|p{4cm}|p{4cm}|}
\hline
\textbf{Caméra} & \textbf{Spectre} & \textbf{Détection} & \textbf{Avantage} \\
\hline
RGB & 400-700 nm & Couleur, forme, texture & Identification visuelle \\
\hline
NIR & 700-1000 nm & Polymères (PE, PP, PET) & Discrimination plastiques \\
\hline
SWIR & 1000-2500 nm & PVC, humidité & Détection chlore \\
\hline
\end{tabular}
\caption{Caractéristiques des caméras multispectrales}
\end{table}

\begin{figure}[H]
\centering
\begin{tikzpicture}
\begin{axis}[
    width=11cm, height=5.5cm,
    xlabel={Longueur d'onde (nm)}, ylabel={Réflectance (\%)},
    xmin=400, xmax=2500, ymin=0, ymax=100,
    legend style={at={(0.98,0.98)}, anchor=north east, font=\tiny},
    grid=major, title={\textbf{Signatures spectrales des matériaux CSR}},
]
\addplot[natranblue, thick, smooth] coordinates {(400,20)(600,25)(800,30)(1000,45)(1400,55)(1800,45)(2200,35)};
\addplot[natranred, thick, smooth] coordinates {(400,15)(600,20)(800,25)(1000,35)(1400,30)(1800,50)(2200,45)};
\addplot[natranorange, thick, smooth] coordinates {(400,60)(600,65)(800,70)(1000,75)(1400,65)(1800,55)(2200,45)};
\addplot[natrangreen!70!black, thick, smooth] coordinates {(400,10)(600,15)(800,25)(1000,35)(1400,50)(1800,50)(2200,40)};
\legend{PE/PP, PVC, Papier, Bois}
\fill[natrangreen!10] (axis cs:400,0) rectangle (axis cs:700,100);
\fill[natranorange!10] (axis cs:700,0) rectangle (axis cs:1000,100);
\fill[natranblue!10] (axis cs:1000,0) rectangle (axis cs:2500,100);
\node[font=\tiny] at (axis cs:550,92) {RGB};
\node[font=\tiny] at (axis cs:850,92) {NIR};
\node[font=\tiny] at (axis cs:1750,92) {SWIR};
\end{axis}
\end{tikzpicture}
\caption{Signatures spectrales des principaux matériaux CSR}
\end{figure}

\subsection{Modèle d'Intelligence Artificielle : YOLOv8}

\subsubsection{Choix du Modèle}

Nous avons choisi \textbf{YOLOv8} pour ses performances en détection d'objets en temps réel.

\begin{table}[H]
\centering
\begin{tabular}{|l|c|c|c|c|}
\hline
\textbf{Modèle} & \textbf{mAP@50} & \textbf{Vitesse (ms)} & \textbf{Paramètres} & \textbf{Adapté CSR} \\
\hline
YOLOv5s & 56.8\% & 6.4 & 7.2M & \cmark \\
YOLOv7 & 51.4\% & 8.7 & 36.9M & \cmark \\
\textbf{YOLOv8m} & \textbf{61.2\%} & \textbf{5.3} & \textbf{25.9M} & \cmark\cmark \\
YOLOv8x & 63.7\% & 12.1 & 68.2M & \cmark \\
\hline
\end{tabular}
\caption{Comparaison des modèles YOLO}
\end{table}

\begin{figure}[H]
\centering
\begin{tikzpicture}[scale=0.6, box/.style={rectangle, draw=natranblue, thick, fill=natranblue!10, minimum width=1.2cm, minimum height=0.6cm, align=center, font=\tiny}]
    \node[box, fill=natrangreen!20] (input) at (0,0) {Image 640×640};
    \node[box] (conv1) at (2,0) {Conv 3×3};
    \node[box] (c2f1) at (4,0) {C2f Block};
    \node[box] (sppf) at (6,0) {SPPF};
    \node[box, fill=natranorange!20] (fpn) at (8,0) {FPN/PAN};
    \node[box, fill=natranred!20] (head) at (10,0) {Detect Head};
    \node[box, fill=natrangreen!20] (output) at (12,0) {Classes + BBox};
    \draw[->, thick] (input) -- (conv1);
    \draw[->, thick] (conv1) -- (c2f1);
    \draw[->, thick] (c2f1) -- (sppf);
    \draw[->, thick] (sppf) -- (fpn);
    \draw[->, thick] (fpn) -- (head);
    \draw[->, thick] (head) -- (output);
    \node[font=\tiny, natrangray] at (3,-1) {Backbone};
    \node[font=\tiny, natrangray] at (8,-1) {Neck};
    \node[font=\tiny, natrangray] at (10,-1) {Head};
\end{tikzpicture}
\caption{Architecture simplifiée de YOLOv8}
\end{figure}

\subsubsection{Performances Simulées}

\begin{figure}[H]
\centering
\begin{tikzpicture}
\begin{axis}[
    width=10cm, height=6cm, ybar, bar width=15pt,
    ylabel={Pourcentage (\%)}, ymin=0, ymax=100,
    symbolic x coords={Précision, Rappel, F1-Score, mAP@50},
    xtick=data, nodes near coords, nodes near coords align={vertical},
    title={\textbf{Performances YOLOv8 vs Tri Manuel}},
    legend style={at={(0.5,-0.2)}, anchor=north},
]
\addplot[fill=natranblue!70] coordinates {(Précision,97.2)(Rappel,95.6)(F1-Score,96.4)(mAP@50,94.8)};
\addplot[fill=natranorange!70] coordinates {(Précision,85)(Rappel,80)(F1-Score,82.4)(mAP@50,78)};
\legend{YOLOv8 (simulé), Tri manuel (réel)}
\end{axis}
\end{tikzpicture}
\caption{Comparaison des performances}
\end{figure}

\subsection{Système Robotique}

\begin{table}[H]
\centering
\begin{tabular}{|l|c|}
\hline
\textbf{Caractéristique} & \textbf{Valeur} \\
\hline
Nombre de bras & 4 (2 accept + 2 reject) \\
Temps de cycle & 200-300 ms \\
Portée & 0.5 m \\
Charge utile & 2 kg \\
Précision & ± 2 mm \\
Durée de vie & 10 millions de cycles \\
\hline
\end{tabular}
\caption{Spécifications des bras robotiques}
\end{table}

\subsection{Système de Contrôle Qualité}

Le système QC est une innovation majeure permettant de corriger les erreurs en temps réel.

\begin{figure}[H]
\centering
\begin{tikzpicture}[scale=0.7]
    \fill[natrangray!30] (-4,2) rectangle (4,2.4);
    \node[font=\tiny] at (0,2.6) {Convoyeur Principal};
    \draw[natranblue, thick, dashed] (-2.5,1.5) rectangle (2.5,2.8);
    \node[font=\tiny, natranblue] at (0,3) {Zone Tri IA};
    \fill[natrangreen!30] (-4,0.5) rectangle (-1.5,0.9);
    \node[font=\tiny] at (-2.75,1.1) {QC Conformes};
    \fill[natranred!30] (1.5,0.5) rectangle (4,0.9);
    \node[font=\tiny] at (2.75,1.1) {QC Non-Conformes};
    \draw[natranorange, very thick] (-2.75,0) -- (-2.75,0.5);
    \node[font=\tiny, natranorange] at (-2.75,-0.3) {Bras QC FP};
    \draw[natrangreen!70!black, very thick] (2.75,0) -- (2.75,0.5);
    \node[font=\tiny, natrangreen!70!black] at (2.75,-0.3) {Bras QC FN};
    \fill[natrangreen!50] (-4,-1.2) rectangle (-2.5,-0.8);
    \node[font=\tiny] at (-3.25,-1) {Réacteur};
    \fill[natranred!50] (2.5,-1.2) rectangle (4,-0.8);
    \node[font=\tiny] at (3.25,-1) {Incinérateur};
\end{tikzpicture}
\caption{Système de contrôle qualité NATRAN}
\end{figure}

\begin{tcolorbox}[colback=natranorange!10, colframe=natranorange, title=Faux Positifs (FP)]
Objets \textbf{non-conformes} classés comme conformes par erreur. Le bras QC orange les éjecte vers le bac de rejet.
\end{tcolorbox}

\begin{tcolorbox}[colback=natrangreen!10, colframe=natrangreen, title=Faux Négatifs (FN)]
Objets \textbf{conformes} classés comme non-conformes par erreur. Le bras QC vert les récupère vers le réacteur.
\end{tcolorbox}

\section{Avantages de Notre Solution}

\subsection{Avantages Techniques}

\begin{table}[H]
\centering
\begin{tabular}{|l|p{5cm}|c|}
\hline
\textbf{Avantage} & \textbf{Description} & \textbf{Impact} \\
\hline
Précision élevée & 97.2\% vs 85\% manuel & +14\% \\
Haute cadence & Jusqu'à 100 obj/min & ×3 \\
Fonctionnement 24/7 & Pas de fatigue ni pause & +40\% productivité \\
Détection PVC & Imagerie SWIR spécifique & Chlore < 0.5\% \\
Contrôle qualité & Correction erreurs temps réel & -80\% erreurs \\
Traçabilité & Logs de chaque objet trié & 100\% traçable \\
\hline
\end{tabular}
\caption{Avantages techniques de NATRAN}
\end{table}

\subsection{Avantages Économiques}

\begin{figure}[H]
\centering
\begin{tikzpicture}
\begin{axis}[
    width=11cm, height=6cm,
    xlabel={Années}, ylabel={Coût cumulé (k€)},
    xmin=0, xmax=5, ymin=0, ymax=1200,
    legend style={at={(0.02,0.98)}, anchor=north west},
    grid=major, title={\textbf{Analyse TCO sur 5 ans}},
]
\addplot[natranred, thick, mark=square] coordinates {(0,0)(1,200)(2,400)(3,600)(4,800)(5,1000)};
\addplot[natranblue, thick, mark=*] coordinates {(0,250)(1,320)(2,390)(3,460)(4,530)(5,600)};
\legend{Tri manuel, NATRAN}
\draw[natrangreen, thick, dashed] (axis cs:1.5,0) -- (axis cs:1.5,350);
\node[font=\tiny, natrangreen] at (axis cs:1.8,380) {ROI 18 mois};
\end{axis}
\end{tikzpicture}
\caption{Analyse du coût total de possession (TCO)}
\end{figure}

\begin{tcolorbox}[colback=natrangreen!10, colframe=natrangreen, title=Économies Réalisées]
\begin{itemize}
    \item \textbf{Réduction main d'œuvre} : -60\% des coûts salariaux
    \item \textbf{Qualité CSR} : +15\% de valeur marchande
    \item \textbf{Pénalités évitées} : -90\% de non-conformités
    \item \textbf{ROI} : 18 mois en moyenne
\end{itemize}
\end{tcolorbox}

\section{Limites et Défis}

\subsection{Limites Techniques}

\begin{table}[H]
\centering
\begin{tabular}{|l|p{6cm}|p{4cm}|}
\hline
\textbf{Limite} & \textbf{Description} & \textbf{Mitigation} \\
\hline
Matériaux noirs & Absorption spectrale limitant la détection & Caméra thermique en option \\
\hline
Objets superposés & Difficulté de séparation & Pré-tri mécanique \\
\hline
Conditions variables & Poussière, humidité & Maintenance préventive \\
\hline
Nouveaux matériaux & Nécessite réentraînement & Mise à jour modèle \\
\hline
\end{tabular}
\caption{Limites techniques et mitigations}
\end{table}

\subsection{Limites Économiques}

\begin{itemize}
    \item \textbf{Investissement initial} : 200-350 k€ selon configuration
    \item \textbf{Compétences requises} : Formation du personnel technique
    \item \textbf{Dépendance technologique} : Maintenance spécialisée
    \item \textbf{Évolutivité} : Coût des mises à jour matérielles
\end{itemize}

\subsection{Défis d'Implémentation}

\begin{figure}[H]
\centering
\begin{tikzpicture}
\begin{axis}[
    width=10cm, height=6cm, xbar, bar width=10pt,
    xlabel={Niveau de difficulté (1-10)},
    symbolic y coords={Formation, Intégration IT, Installation, Calibration, Maintenance},
    ytick=data, nodes near coords,
    title={\textbf{Défis d'implémentation}},
    xmin=0, xmax=10,
]
\addplot[fill=natranorange!70] coordinates {(7,Formation)(6,Intégration IT)(5,Installation)(8,Calibration)(4,Maintenance)};
\end{axis}
\end{tikzpicture}
\caption{Évaluation des défis d'implémentation}
\end{figure}

\section{Types de CSR Traités}

\begin{table}[H]
\centering
\begin{tabular}{|l|l|c|c|c|c|}
\hline
\textbf{Type} & \textbf{Nom complet} & \textbf{Décision} & \textbf{PCI} & \textbf{Cl (\%)} & \textbf{Danger} \\
\hline
PE/PP & Polyéthylène/Polypropylène & \textcolor{natrangreen}{Accept} & 43 & 0 & Non \\
Carton & Carton/Papier souillé & \textcolor{natrangreen}{Accept} & 16 & 0 & Non \\
Bois & Bois traité/Palettes & \textcolor{natrangreen}{Accept} & 17 & 0 & Non \\
Textile & Textile non-chloré & \textcolor{natrangreen}{Accept} & 20 & 0 & Non \\
\hline
PVC & Polychlorure de vinyle & \textcolor{natranred}{Reject} & 18 & 57 & \textbf{Oui} \\
Métal & Métaux ferreux/non-ferreux & \textcolor{natranred}{Reject} & 5 & 0 & \textbf{Oui} \\
Verre & Verre/Céramique & \textcolor{natranred}{Reject} & 0 & 0 & Non \\
Caoutchouc & Caoutchouc/Pneus & \textcolor{natranred}{Reject} & 32 & 15 & \textbf{Oui} \\
\hline
PET & Polyéthylène téréphtalate & \textcolor{natranorange}{Incertain} & 23 & 0 & Non \\
Multi-couche & Emballage complexe & \textcolor{natranorange}{Incertain} & 25 & 5 & \textbf{Oui} \\
\hline
\end{tabular}
\caption{Classification des types de CSR traités par NATRAN}
\end{table}

\begin{figure}[H]
\centering
\begin{tikzpicture}
\begin{axis}[
    width=11cm, height=6cm,
    xlabel={PCI (MJ/kg)}, ylabel={Chlore (\%)},
    xmin=0, xmax=50, ymin=0, ymax=60,
    grid=major, title={\textbf{Cartographie des matériaux CSR}},
    legend style={at={(0.98,0.98)}, anchor=north east, font=\tiny},
]
% Zone acceptable
\fill[natrangreen!20] (axis cs:15,0) rectangle (axis cs:50,1);
% Zone rejet
\fill[natranred!20] (axis cs:0,1) rectangle (axis cs:50,60);
\fill[natranred!20] (axis cs:0,0) rectangle (axis cs:15,60);

% Points
\addplot[only marks, mark=*, mark size=3pt, natrangreen] coordinates {(43,0)(16,0)(17,0)(20,0)};
\addplot[only marks, mark=square*, mark size=3pt, natranred] coordinates {(18,57)(5,0)(0,0)(32,15)};
\addplot[only marks, mark=triangle*, mark size=3pt, natranorange] coordinates {(23,0)(25,5)};

\legend{Conformes, Non-conformes, Incertains}

% Labels
\node[font=\tiny] at (axis cs:43,3) {PE/PP};
\node[font=\tiny] at (axis cs:18,54) {PVC};
\node[font=\tiny] at (axis cs:32,18) {Caoutchouc};
\end{axis}
\end{tikzpicture}
\caption{Cartographie PCI/Chlore des matériaux CSR}
\end{figure}
% ============================================
% PARTIE 3 : COMPARATIF ET ÉVALUATION
% ============================================
\newpage
\part{Comparatif et Évaluation}

\section{Méthodologie de Comparaison}

\subsection{Critères d'Évaluation}

Pour comparer objectivement NATRAN aux solutions existantes, nous avons défini 8 critères pondérés :

\begin{table}[H]
\centering
\begin{tabular}{|l|c|p{6cm}|}
\hline
\textbf{Critère} & \textbf{Poids} & \textbf{Description} \\
\hline
Précision de tri & 20\% & Taux de classification correcte \\
Cadence & 15\% & Nombre d'objets traités par minute \\
Détection PVC & 15\% & Capacité à identifier le chlore \\
Coût d'investissement & 15\% & Investissement initial requis \\
Coût opérationnel & 10\% & Coûts de fonctionnement annuels \\
Maintenance & 10\% & Facilité et coût de maintenance \\
Adaptabilité & 10\% & Capacité à traiter différents flux \\
Traçabilité & 5\% & Suivi et reporting des opérations \\
\hline
\end{tabular}
\caption{Critères d'évaluation et pondération}
\end{table}

\subsection{Solutions Comparées}

\begin{enumerate}
    \item \textbf{Tri manuel} : Référence actuelle du marché
    \item \textbf{Tri optique NIR} : Technologie mature (TOMRA, Pellenc ST)
    \item \textbf{Tri rayons X} : Haute précision (Steinert, Sesotec)
    \item \textbf{IA concurrente} : Solutions existantes (ZenRobotics, AMP Robotics)
    \item \textbf{NATRAN} : Notre solution
\end{enumerate}

\section{Comparaison Détaillée}

\subsection{Performances de Tri}

\begin{figure}[H]
\centering
\begin{tikzpicture}
\begin{axis}[
    width=12cm, height=7cm, ybar, bar width=10pt,
    ylabel={Précision (\%)}, ymin=70, ymax=100,
    symbolic x coords={Manuel, NIR, Rayons X, IA Concur., NATRAN},
    xtick=data, nodes near coords, nodes near coords align={vertical},
    title={\textbf{Comparaison de la précision de tri}},
    legend style={at={(0.5,-0.2)}, anchor=north},
]
\addplot[fill=natrangray!70] coordinates {(Manuel,85)(NIR,90)(Rayons X,92)(IA Concur.,88)(NATRAN,97.2)};
\end{axis}
\end{tikzpicture}
\caption{Précision de tri par technologie}
\end{figure}

\begin{figure}[H]
\centering
\begin{tikzpicture}
\begin{axis}[
    width=12cm, height=7cm, ybar, bar width=10pt,
    ylabel={Objets/minute}, ymin=0, ymax=120,
    symbolic x coords={Manuel, NIR, Rayons X, IA Concur., NATRAN},
    xtick=data, nodes near coords, nodes near coords align={vertical},
    title={\textbf{Comparaison de la cadence de tri}},
]
\addplot[fill=natranblue!70] coordinates {(Manuel,25)(NIR,80)(Rayons X,60)(IA Concur.,70)(NATRAN,100)};
\end{axis}
\end{tikzpicture}
\caption{Cadence de tri par technologie}
\end{figure}

\subsection{Détection du PVC (Chlore)}

La détection du PVC est critique pour la qualité des CSR. Le chlore génère de l'acide chlorhydrique (HCl) lors de la combustion.

\begin{table}[H]
\centering
\begin{tabular}{|l|c|c|c|}
\hline
\textbf{Technologie} & \textbf{Détection PVC} & \textbf{Taux réussite} & \textbf{Méthode} \\
\hline
Tri manuel & Limitée & 70\% & Visuelle \\
NIR & Partielle & 75\% & Absorption spectrale \\
Rayons X & Bonne & 88\% & Densité atomique \\
IA Concurrente & Variable & 80\% & RGB uniquement \\
\textbf{NATRAN} & \textbf{Excellente} & \textbf{95\%} & \textbf{SWIR + IA} \\
\hline
\end{tabular}
\caption{Comparaison de la détection du PVC}
\end{table}

\begin{figure}[H]
\centering
\begin{tikzpicture}
\begin{axis}[
    width=10cm, height=6cm,
    xlabel={Taux de détection PVC (\%)}, ylabel={Taux de chlore résiduel (\%)},
    xmin=60, xmax=100, ymin=0, ymax=2,
    grid=major, title={\textbf{Impact de la détection PVC sur le chlore résiduel}},
]
\addplot[natranred, thick, smooth, mark=*] coordinates {(70,1.8)(75,1.4)(80,1.0)(85,0.7)(88,0.5)(95,0.2)};
\draw[natrangreen, thick, dashed] (axis cs:60,1) -- (axis cs:100,1);
\node[font=\tiny, natrangreen] at (axis cs:65,1.15) {Seuil réglementaire};
\draw[natranblue, thick, dashed] (axis cs:95,0) -- (axis cs:95,2);
\node[font=\tiny, natranblue, rotate=90] at (axis cs:96.5,1) {NATRAN};
\end{axis}
\end{tikzpicture}
\caption{Corrélation détection PVC / chlore résiduel}
\end{figure}

\subsection{Analyse des Coûts}

\subsubsection{Investissement Initial}

\begin{figure}[H]
\centering
\begin{tikzpicture}
\begin{axis}[
    width=11cm, height=6cm, xbar, bar width=12pt,
    xlabel={Investissement (k€)},
    symbolic y coords={Manuel, NATRAN, IA Concur., NIR, Rayons X},
    ytick=data, nodes near coords, nodes near coords align={horizontal},
    title={\textbf{Investissement initial par technologie}},
    xmin=0, xmax=900,
]
\addplot[fill=natranblue!70] coordinates {(50,Manuel)(280,NATRAN)(350,IA Concur.)(400,NIR)(650,Rayons X)};
\end{axis}
\end{tikzpicture}
\caption{Comparaison des investissements initiaux}
\end{figure}

\subsubsection{Coûts Opérationnels Annuels}

\begin{table}[H]
\centering
\begin{tabular}{|l|c|c|c|c|}
\hline
\textbf{Technologie} & \textbf{Main d'œuvre} & \textbf{Énergie} & \textbf{Maintenance} & \textbf{Total/an} \\
\hline
Tri manuel & 180 k€ & 5 k€ & 10 k€ & \textbf{195 k€} \\
NIR & 40 k€ & 15 k€ & 35 k€ & \textbf{90 k€} \\
Rayons X & 40 k€ & 25 k€ & 45 k€ & \textbf{110 k€} \\
IA Concurrente & 50 k€ & 12 k€ & 30 k€ & \textbf{92 k€} \\
\textbf{NATRAN} & \textbf{35 k€} & \textbf{10 k€} & \textbf{25 k€} & \textbf{70 k€} \\
\hline
\end{tabular}
\caption{Coûts opérationnels annuels par technologie}
\end{table}

\subsubsection{Retour sur Investissement}

\begin{figure}[H]
\centering
\begin{tikzpicture}
\begin{axis}[
    width=12cm, height=7cm,
    xlabel={Années}, ylabel={Coût cumulé (k€)},
    xmin=0, xmax=5, ymin=0, ymax=1200,
    legend style={at={(0.02,0.98)}, anchor=north west, font=\small},
    grid=major, title={\textbf{Évolution du coût total sur 5 ans}},
]
\addplot[natranred, thick, mark=square] coordinates {(0,50)(1,245)(2,440)(3,635)(4,830)(5,1025)};
\addplot[natranorange, thick, mark=triangle] coordinates {(0,400)(1,490)(2,580)(3,670)(4,760)(5,850)};
\addplot[natrangray, thick, mark=diamond] coordinates {(0,650)(1,760)(2,870)(3,980)(4,1090)(5,1200)};
\addplot[natrangreen!70!black, thick, mark=pentagon] coordinates {(0,350)(1,442)(2,534)(3,626)(4,718)(5,810)};
\addplot[natranblue, thick, mark=*, line width=1.5pt] coordinates {(0,280)(1,350)(2,420)(3,490)(4,560)(5,630)};
\legend{Manuel, NIR, Rayons X, IA Concur., NATRAN}
\end{axis}
\end{tikzpicture}
\caption{Analyse TCO comparative sur 5 ans}
\end{figure}

\begin{tcolorbox}[colback=natrangreen!10, colframe=natrangreen, title=Analyse ROI NATRAN]
\begin{itemize}
    \item \textbf{Point mort vs Manuel} : 14 mois
    \item \textbf{Économies sur 5 ans} : 395 k€ vs tri manuel
    \item \textbf{ROI annuel moyen} : 28\%
\end{itemize}
\end{tcolorbox}

\section{Tableau Comparatif Global}

\begin{table}[H]
\centering
\small
\begin{tabular}{|l|c|c|c|c|c|}
\hline
\textbf{Critère} & \textbf{Manuel} & \textbf{NIR} & \textbf{Rayons X} & \textbf{IA Conc.} & \textbf{NATRAN} \\
\hline
Précision & 85\% & 90\% & 92\% & 88\% & \textbf{97.2\%} \\
Cadence (obj/min) & 25 & 80 & 60 & 70 & \textbf{100} \\
Détection PVC & 70\% & 75\% & 88\% & 80\% & \textbf{95\%} \\
Investissement & 50 k€ & 400 k€ & 650 k€ & 350 k€ & \textbf{280 k€} \\
Coût annuel & 195 k€ & 90 k€ & 110 k€ & 92 k€ & \textbf{70 k€} \\
Maintenance & Simple & Complexe & Très complexe & Moyenne & \textbf{Moyenne} \\
Adaptabilité & Haute & Moyenne & Faible & Haute & \textbf{Haute} \\
Traçabilité & Faible & Moyenne & Moyenne & Haute & \textbf{Haute} \\
\hline
\textbf{Score global} & 5.2/10 & 6.8/10 & 7.1/10 & 7.0/10 & \textbf{8.9/10} \\
\hline
\end{tabular}
\caption{Tableau comparatif global des technologies de tri}
\end{table}

\section{Analyse Radar Multicritère}

\begin{figure}[H]
\centering
\begin{tikzpicture}[scale=1.2]
    % Axes
    \foreach \i/\label in {1/Précision, 2/Cadence, 3/PVC, 4/Coût inv., 5/Coût op., 6/Maintenance, 7/Adaptabilité, 8/Traçabilité} {
        \draw[natrangray] (0,0) -- ({90-(\i-1)*45}:3);
        \node[font=\tiny] at ({90-(\i-1)*45}:3.4) {\label};
    }
    
    % Grille
    \foreach \r in {1,2,3} {
        \draw[natrangray!30] ({90}:\r) \foreach \i in {2,...,8} { -- ({90-(\i-1)*45}:\r)} -- cycle;
    }
    
    % Manuel (rouge)
    \draw[natranred, thick, fill=natranred!20, opacity=0.5] 
        (90:1.7) -- (45:0.5) -- (0:1.4) -- (-45:2.9) -- (-90:2.9) -- (-135:2.7) -- (180:2.7) -- (135:0.6) -- cycle;
    
    % NATRAN (bleu)
    \draw[natranblue, thick, fill=natranblue!20, opacity=0.7] 
        (90:2.9) -- (45:3) -- (0:2.85) -- (-45:2.2) -- (-90:2.1) -- (-135:2.1) -- (180:2.7) -- (135:2.7) -- cycle;
    
    % Légende
    \node[font=\small, natranred] at (-3,-2) {--- Manuel};
    \node[font=\small, natranblue] at (3,-2) {--- NATRAN};
\end{tikzpicture}
\caption{Analyse radar : NATRAN vs Tri manuel}
\end{figure}

\section{Résultats de Simulation}

\subsection{Protocole de Test}

La simulation a été réalisée avec les paramètres suivants :
\begin{itemize}
    \item \textbf{Durée} : Sessions de 30 minutes
    \item \textbf{Flux} : 100 objets/minute
    \item \textbf{Composition} : 40\% conformes, 50\% non-conformes, 10\% incertains
    \item \textbf{Probabilité FP} : 2.8\% (basé sur études YOLOv8)
    \item \textbf{Probabilité FN} : 4.4\% (basé sur études YOLOv8)
\end{itemize}

\subsection{Résultats Obtenus}

\begin{table}[H]
\centering
\begin{tabular}{|l|c|c|}
\hline
\textbf{Métrique} & \textbf{Objectif} & \textbf{Résultat Simulation} \\
\hline
Précision & $\geq$ 95\% & \textbf{97.2\%} \cmark \\
Rappel & $\geq$ 95\% & \textbf{95.6\%} \cmark \\
F1-Score & $\geq$ 95\% & \textbf{96.4\%} \cmark \\
Taux de chlore & $<$ 1\% & \textbf{0.3\%} \cmark \\
Qualité syngas & $>$ 90\% & \textbf{94.5\%} \cmark \\
Cadence & $>$ 80 obj/min & \textbf{100 obj/min} \cmark \\
\hline
\end{tabular}
\caption{Résultats de la simulation NATRAN}
\end{table}

\begin{figure}[H]
\centering
\begin{tikzpicture}
\begin{axis}[
    width=11cm, height=6cm,
    xlabel={Temps (minutes)}, ylabel={Taux (\%)},
    xmin=0, xmax=30, ymin=90, ymax=100,
    legend style={at={(0.98,0.02)}, anchor=south east, font=\small},
    grid=major, title={\textbf{Évolution des métriques pendant la simulation}},
]
\addplot[natranblue, thick] coordinates {(0,95)(5,96.5)(10,97)(15,97.1)(20,97.2)(25,97.2)(30,97.2)};
\addplot[natrangreen, thick] coordinates {(0,93)(5,94.5)(10,95.2)(15,95.5)(20,95.6)(25,95.6)(30,95.6)};
\addplot[natranorange, thick] coordinates {(0,94)(5,95.5)(10,96)(15,96.2)(20,96.4)(25,96.4)(30,96.4)};
\legend{Précision, Rappel, F1-Score}
\end{axis}
\end{tikzpicture}
\caption{Convergence des métriques pendant la simulation}
\end{figure}

\subsection{Matrice de Confusion}

\begin{figure}[H]
\centering
\begin{tikzpicture}[scale=0.9]
    % Matrice
    \draw[thick] (0,0) rectangle (6,6);
    \draw[thick] (3,0) -- (3,6);
    \draw[thick] (0,3) -- (6,3);
    
    % Headers
    \node[font=\bfseries] at (3,6.5) {Prédit};
    \node[font=\bfseries, rotate=90] at (-0.5,3) {Réel};
    \node[font=\small] at (1.5,6.3) {Conforme};
    \node[font=\small] at (4.5,6.3) {Non-conforme};
    \node[font=\small, rotate=90] at (-0.3,4.5) {Conforme};
    \node[font=\small, rotate=90] at (-0.3,1.5) {Non-conforme};
    
    % Cellules
    \fill[natrangreen!30] (0,3) rectangle (3,6);
    \fill[natranred!30] (3,3) rectangle (6,6);
    \fill[natranred!30] (0,0) rectangle (3,3);
    \fill[natrangreen!30] (3,0) rectangle (6,3);
    
    % Valeurs
    \node[font=\Large\bfseries] at (1.5,4.5) {956};
    \node[font=\small] at (1.5,3.8) {Vrais Positifs};
    
    \node[font=\Large\bfseries] at (4.5,4.5) {44};
    \node[font=\small] at (4.5,3.8) {Faux Négatifs};
    
    \node[font=\Large\bfseries] at (1.5,1.5) {28};
    \node[font=\small] at (1.5,0.8) {Faux Positifs};
    
    \node[font=\Large\bfseries] at (4.5,1.5) {972};
    \node[font=\small] at (4.5,0.8) {Vrais Négatifs};
\end{tikzpicture}
\caption{Matrice de confusion (sur 2000 objets simulés)}
\end{figure}

\section{Avantages Concurrentiels de NATRAN}

\begin{tcolorbox}[colback=natranblue!5, colframe=natranblue, title=Différenciateurs Clés]
\begin{enumerate}
    \item \textbf{Imagerie SWIR} : Seule solution combinant RGB + NIR + SWIR pour une détection optimale du PVC
    \item \textbf{Contrôle Qualité intégré} : Correction automatique des erreurs de classification
    \item \textbf{Coût optimisé} : 30\% moins cher que les solutions NIR traditionnelles
    \item \textbf{Modularité} : Architecture évolutive et personnalisable
    \item \textbf{Traçabilité complète} : Logs détaillés de chaque objet trié
\end{enumerate}
\end{tcolorbox}
% ============================================
% PARTIE 4 : CONCLUSION
% ============================================
\newpage
\part{Conclusion et Perspectives}

\section{Synthèse des Résultats}

\subsection{Objectifs Atteints}

Ce projet a permis de démontrer la faisabilité d'un système de tri automatisé de CSR basé sur l'intelligence artificielle. Les principaux objectifs ont été atteints :

\begin{table}[H]
\centering
\begin{tabular}{|l|c|c|c|}
\hline
\textbf{Objectif} & \textbf{Cible} & \textbf{Résultat} & \textbf{Statut} \\
\hline
Précision de tri & $\geq$ 95\% & 97.2\% & \cmark \\
Détection PVC & $\geq$ 90\% & 95\% & \cmark \\
Cadence & $\geq$ 80 obj/min & 100 obj/min & \cmark \\
Taux de chlore & $<$ 1\% & 0.3\% & \cmark \\
ROI & $<$ 24 mois & 18 mois & \cmark \\
Simulation 3D & Fonctionnelle & Opérationnelle & \cmark \\
\hline
\end{tabular}
\caption{Bilan des objectifs du projet}
\end{table}

\subsection{Contributions Principales}

\begin{enumerate}
    \item \textbf{Analyse de marché approfondie} : État des lieux complet du secteur des CSR en France et en Europe, identification des besoins non satisfaits et des opportunités.
    
    \item \textbf{Architecture innovante} : Conception d'un système combinant imagerie multispectrale (RGB + NIR + SWIR), intelligence artificielle (YOLOv8) et contrôle qualité intégré.
    
    \item \textbf{Simulation 3D interactive} : Développement d'une plateforme de démonstration permettant de visualiser le fonctionnement du système en temps réel.
    
    \item \textbf{Évaluation comparative} : Analyse rigoureuse des performances par rapport aux solutions existantes, démontrant les avantages de NATRAN.
\end{enumerate}

\subsection{Apports Scientifiques et Techniques}

\begin{figure}[H]
\centering
\begin{tikzpicture}[scale=0.8]
    % Cercles concentriques
    \fill[natranblue!10] (0,0) circle (4);
    \fill[natranblue!20] (0,0) circle (3);
    \fill[natranblue!30] (0,0) circle (2);
    \fill[natranblue!50] (0,0) circle (1);
    
    % Labels
    \node[font=\small\bfseries, white] at (0,0) {IA};
    \node[font=\tiny] at (0,1.5) {Vision};
    \node[font=\tiny] at (0,2.5) {Robotique};
    \node[font=\tiny] at (0,3.5) {Industrie};
    
    % Flèches
    \draw[->, thick, natrangreen] (4.5,0) -- (5.5,0);
    \node[font=\small, align=left] at (7.5,0) {Impact\\croissant};
\end{tikzpicture}
\caption{Niveaux d'impact du projet NATRAN}
\end{figure}

\section{Limites et Axes d'Amélioration}

\subsection{Limites de l'Étude}

\begin{tcolorbox}[colback=natranorange!10, colframe=natranorange, title=Limites Identifiées]
\begin{enumerate}
    \item \textbf{Simulation vs Réalité} : Les performances simulées sont basées sur des études académiques et non sur des tests en conditions réelles avec de vrais CSR.
    
    \item \textbf{Variabilité des flux} : La simulation utilise une composition fixe (40\% conformes, 50\% non-conformes, 10\% incertains) qui peut différer des flux réels.
    
    \item \textbf{Conditions environnementales} : Les effets de la poussière, de l'humidité et de l'éclairage variable n'ont pas été modélisés.
    
    \item \textbf{Coûts estimatifs} : Les coûts présentés sont des estimations basées sur des données de marché et peuvent varier selon les fournisseurs.
\end{enumerate}
\end{tcolorbox}

\subsection{Axes d'Amélioration}

\begin{table}[H]
\centering
\begin{tabular}{|l|p{5cm}|c|}
\hline
\textbf{Axe} & \textbf{Description} & \textbf{Priorité} \\
\hline
Entraînement réel & Collecter des données CSR réelles pour entraîner YOLOv8 & Haute \\
\hline
Tests terrain & Valider les performances sur site industriel & Haute \\
\hline
Optimisation vitesse & Réduire la latence de détection à < 30ms & Moyenne \\
\hline
Détection thermique & Ajouter une caméra thermique pour les matériaux noirs & Moyenne \\
\hline
Interface opérateur & Développer un dashboard de supervision avancé & Basse \\
\hline
\end{tabular}
\caption{Axes d'amélioration prioritaires}
\end{table}

\section{Perspectives}

\subsection{Court Terme (6-12 mois)}

\begin{itemize}
    \item \textbf{Validation industrielle} : Partenariat avec un exploitant de CSR pour tests en conditions réelles
    \item \textbf{Entraînement du modèle} : Constitution d'un dataset de CSR réels et fine-tuning de YOLOv8
    \item \textbf{Certification} : Obtention des certifications CE et conformité réglementaire
\end{itemize}

\subsection{Moyen Terme (1-2 ans)}

\begin{itemize}
    \item \textbf{Commercialisation} : Lancement commercial auprès des PME du secteur
    \item \textbf{Évolutions techniques} : Intégration de nouvelles modalités de détection
    \item \textbf{Expansion géographique} : Déploiement en Europe (Allemagne, Italie, Espagne)
\end{itemize}

\subsection{Long Terme (3-5 ans)}

\begin{itemize}
    \item \textbf{Plateforme SaaS} : Offre de tri en tant que service avec maintenance prédictive
    \item \textbf{Diversification} : Extension à d'autres flux de déchets (DEEE, plastiques, textiles)
    \item \textbf{Intelligence collective} : Apprentissage fédéré entre installations pour amélioration continue
\end{itemize}

\begin{figure}[H]
\centering
\begin{tikzpicture}[scale=0.8]
    % Timeline
    \draw[->, very thick, natranblue] (0,0) -- (14,0);
    
    % Années
    \foreach \x/\year in {0/2025, 3/2026, 6/2027, 9/2028, 12/2029} {
        \draw[thick] (\x,-0.2) -- (\x,0.2);
        \node[font=\small] at (\x,-0.5) {\year};
    }
    
    % Jalons
    \fill[natrangreen] (1,0) circle (0.2);
    \node[font=\tiny, align=center] at (1,1) {Simulation\\terminée};
    
    \fill[natranorange] (3,0) circle (0.2);
    \node[font=\tiny, align=center] at (3,1) {Tests\\industriels};
    
    \fill[natranblue] (5,0) circle (0.2);
    \node[font=\tiny, align=center] at (5,1) {Lancement\\commercial};
    
    \fill[natranblue] (8,0) circle (0.2);
    \node[font=\tiny, align=center] at (8,1) {Expansion\\Europe};
    
    \fill[natranblue] (11,0) circle (0.2);
    \node[font=\tiny, align=center] at (11,1) {Plateforme\\SaaS};
\end{tikzpicture}
\caption{Roadmap du projet NATRAN}
\end{figure}

\section{Conclusion Générale}

Le projet NATRAN démontre le potentiel de l'intelligence artificielle pour révolutionner le tri des Combustibles Solides de Récupération. Face aux limites du tri manuel (coût, précision, sécurité), notre solution propose une alternative performante et économiquement viable.

\vspace{0.5cm}

Les résultats de simulation sont encourageants :
\begin{itemize}
    \item \textbf{Précision de 97.2\%} contre 85\% pour le tri manuel
    \item \textbf{Détection du PVC à 95\%} grâce à l'imagerie SWIR
    \item \textbf{ROI de 18 mois} avec des économies de 395 k€ sur 5 ans
    \item \textbf{Contrôle qualité intégré} réduisant les erreurs de 80\%
\end{itemize}

\vspace{0.5cm}

\begin{tcolorbox}[colback=natranblue!5, colframe=natranblue]
\centering
\textbf{NATRAN représente une avancée significative vers l'automatisation intelligente du tri des déchets industriels, contribuant à l'économie circulaire et à la transition énergétique.}
\end{tcolorbox}

\vspace{1cm}

% ============================================
% DISCLAIMER
% ============================================
\section*{Disclaimer}
\addcontentsline{toc}{section}{Disclaimer}

\begin{tcolorbox}[colback=natranred!5, colframe=natranred, title=\faExclamationTriangle\ Note Importante sur les Données]

\textbf{Concernant les chiffres présentés dans ce rapport :}

\vspace{0.3cm}

\textbf{1. Données du tri manuel (réelles) :}
\begin{itemize}
    \item Les performances du tri manuel (précision 80-85\%, cadence 20-30 obj/min, coûts) sont issues d'études industrielles et académiques publiées.
    \item Sources : ADEME, FEDEREC, études sectorielles européennes.
\end{itemize}

\vspace{0.3cm}

\textbf{2. Données de la simulation NATRAN :}
\begin{itemize}
    \item Les performances simulées (précision 97.2\%, rappel 95.6\%) sont basées sur les performances documentées de YOLOv8 sur des tâches de détection d'objets similaires.
    \item \textbf{Ces chiffres ne proviennent PAS de tests sur de vrais CSR}, car nous n'avons pas eu accès à un flux réel de Combustibles Solides de Récupération.
    \item Les probabilités de faux positifs (2.8\%) et faux négatifs (4.4\%) sont extrapolées à partir d'études sur YOLOv8 appliqué à la détection de déchets.
\end{itemize}

\vspace{0.3cm}

\textbf{3. Implications :}
\begin{itemize}
    \item Les résultats présentés sont des \textbf{projections théoriques} basées sur l'état de l'art.
    \item Une validation en conditions réelles serait nécessaire avant tout déploiement industriel.
    \item Les performances réelles pourraient varier en fonction de la qualité des images, de la variabilité des matériaux et des conditions environnementales.
\end{itemize}

\end{tcolorbox}

% ============================================
% RÉFÉRENCES
% ============================================
\newpage
\section*{Références}
\addcontentsline{toc}{section}{Références}

\begin{enumerate}
    \item ADEME (2023). \textit{État des lieux de la filière CSR en France}. Rapport annuel.
    
    \item FEDEREC (2023). \textit{Rapport d'activité de la fédération des entreprises du recyclage}.
    
    \item Ultralytics (2023). \textit{YOLOv8: State-of-the-art object detection}. Documentation technique.
    
    \item European Commission (2022). \textit{Solid Recovered Fuels - Specifications and classes}. EN ISO 21640.
    
    \item Wang, C. et al. (2023). \textit{Deep learning for waste sorting: A comprehensive review}. Waste Management, 150, 234-251.
    
    \item TOMRA Systems (2022). \textit{Sensor-based sorting technologies for waste management}. White paper.
    
    \item Zheng, Y. et al. (2022). \textit{NIR spectroscopy for plastic identification in waste streams}. Resources, Conservation and Recycling, 178, 106023.
    
    \item Ministère de la Transition Écologique (2023). \textit{Feuille de route économie circulaire}.
    
    \item Babylon.js Documentation (2024). \textit{3D engine for web applications}. https://doc.babylonjs.com
    
    \item React Documentation (2024). \textit{A JavaScript library for building user interfaces}. https://react.dev
\end{enumerate}

% ============================================
% ANNEXES
% ============================================
\newpage
\section*{Annexes}
\addcontentsline{toc}{section}{Annexes}

\subsection*{A. Liens du Projet}

\begin{itemize}
    \item \textbf{GitHub} : \url{https://github.com/selyan-mhli/projet_Natran}
    \item \textbf{Documentation technique} : Disponible dans le repository
\end{itemize}

\subsection*{B. Structure du Code Source}

\begin{verbatim}
projet_Natran/
├── src/
│   ├── components/
│   │   ├── FinalSimulation.tsx   # Simulation 3D Babylon.js
│   │   ├── Dashboard.tsx         # Monitoring temps réel
│   │   ├── Architecture.tsx      # Schéma du système
│   │   ├── Impact.tsx            # Page impact environnemental
│   │   └── Header.tsx            # Navigation
│   ├── context/
│   │   └── SimulationContext.tsx # État global (stats)
│   └── App.tsx                   # Point d'entrée
├── index.html
├── package.json
├── vite.config.ts
└── tailwind.config.js
\end{verbatim}

\subsection*{C. Technologies Utilisées}

\begin{table}[H]
\centering
\begin{tabular}{|l|l|l|}
\hline
\textbf{Catégorie} & \textbf{Technologie} & \textbf{Version} \\
\hline
Framework & React & 18.x \\
Langage & TypeScript & 5.x \\
3D Engine & Babylon.js & 6.x \\
Styling & TailwindCSS & 3.x \\
Build & Vite & 5.x \\
Graphiques & Recharts & 2.x \\
Icônes & Lucide React & 0.x \\
\hline
\end{tabular}
\caption{Stack technique du projet}
\end{table}

\subsection*{D. Glossaire}

\begin{description}
    \item[CSR] Combustible Solide de Récupération
    \item[PCI] Pouvoir Calorifique Inférieur (MJ/kg)
    \item[PVC] Polychlorure de Vinyle
    \item[NIR] Near Infrared (Proche Infrarouge)
    \item[SWIR] Short-Wave Infrared (Infrarouge à Ondes Courtes)
    \item[YOLOv8] You Only Look Once version 8 (modèle de détection)
    \item[mAP] mean Average Precision
    \item[FP] Faux Positif
    \item[FN] Faux Négatif
    \item[TP] Vrai Positif (True Positive)
    \item[TN] Vrai Négatif (True Negative)
    \item[ROI] Return On Investment
    \item[TCO] Total Cost of Ownership
\end{description}

\end{document}
