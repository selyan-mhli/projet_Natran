% ============================================
% PARTIE 2 : NOTRE SOLUTION
% ============================================
\newpage
\part{Notre Solution : NATRAN}

\section{Présentation Générale}

\subsection{Vision et Objectifs}

\textbf{NATRAN} (Natural Transformation) est un système de tri intelligent combinant :
\begin{itemize}
    \item Vision par ordinateur avec caméras multispectrales
    \item Intelligence artificielle (YOLOv8)
    \item Robotique automatisée
    \item Contrôle qualité intégré
\end{itemize}

\begin{tcolorbox}[colback=natranblue!5, colframe=natranblue, title=Notre Vision]
\textbf{Démocratiser l'accès au tri automatisé de haute précision pour tous les acteurs de la filière CSR.}
\end{tcolorbox}

\subsection{Architecture du Système}

\begin{figure}[H]
\centering
\begin{tikzpicture}[scale=0.65, node distance=0.8cm,
    box/.style={rectangle, draw=natranblue, thick, fill=natranblue!10, minimum width=2.2cm, minimum height=0.9cm, align=center, font=\tiny}]
    
    \node[box, fill=natrangreen!20] (cam1) at (-3,3) {Caméra RGB};
    \node[box, fill=natrangreen!20] (cam2) at (0,3) {Caméra NIR};
    \node[box, fill=natrangreen!20] (cam3) at (3,3) {Caméra SWIR};
    \node[box, fill=natranorange!20] (fusion) at (0,1.5) {Fusion Multimodale};
    \node[box, fill=natranorange!20] (yolo) at (0,0) {YOLOv8 Détection};
    \node[box, fill=natranblue!20] (decision) at (0,-1.5) {Moteur Décision};
    \node[box, fill=natranred!20] (robot1) at (-2.5,-3) {Bras Accept};
    \node[box, fill=natranred!20] (robot2) at (2.5,-3) {Bras Reject};
    \node[box, fill=natranyellow!20] (qc) at (0,-4.5) {Contrôle Qualité};
    
    \draw[->, thick, natranblue] (cam1) -- (fusion);
    \draw[->, thick, natranblue] (cam2) -- (fusion);
    \draw[->, thick, natranblue] (cam3) -- (fusion);
    \draw[->, thick, natranblue] (fusion) -- (yolo);
    \draw[->, thick, natranblue] (yolo) -- (decision);
    \draw[->, thick, natranblue] (decision) -- (robot1);
    \draw[->, thick, natranblue] (decision) -- (robot2);
    \draw[->, thick, natranblue] (robot1) |- (qc);
    \draw[->, thick, natranblue] (robot2) |- (qc);
\end{tikzpicture}
\caption{Architecture en couches du système NATRAN}
\end{figure}

\section{Composants Techniques}

\subsection{Système de Vision Multispectrale}

\begin{table}[H]
\centering
\begin{tabular}{|l|c|p{4cm}|p{4cm}|}
\hline
\textbf{Caméra} & \textbf{Spectre} & \textbf{Détection} & \textbf{Avantage} \\
\hline
RGB & 400-700 nm & Couleur, forme, texture & Identification visuelle \\
\hline
NIR & 700-1000 nm & Polymères (PE, PP, PET) & Discrimination plastiques \\
\hline
SWIR & 1000-2500 nm & PVC, humidité & Détection chlore \\
\hline
\end{tabular}
\caption{Caractéristiques des caméras multispectrales}
\end{table}

\begin{figure}[H]
\centering
\begin{tikzpicture}
\begin{axis}[
    width=11cm, height=5.5cm,
    xlabel={Longueur d'onde (nm)}, ylabel={Réflectance (\%)},
    xmin=400, xmax=2500, ymin=0, ymax=100,
    legend style={at={(0.98,0.98)}, anchor=north east, font=\tiny},
    grid=major, title={\textbf{Signatures spectrales des matériaux CSR}},
]
\addplot[natranblue, thick, smooth] coordinates {(400,20)(600,25)(800,30)(1000,45)(1400,55)(1800,45)(2200,35)};
\addplot[natranred, thick, smooth] coordinates {(400,15)(600,20)(800,25)(1000,35)(1400,30)(1800,50)(2200,45)};
\addplot[natranorange, thick, smooth] coordinates {(400,60)(600,65)(800,70)(1000,75)(1400,65)(1800,55)(2200,45)};
\addplot[natrangreen!70!black, thick, smooth] coordinates {(400,10)(600,15)(800,25)(1000,35)(1400,50)(1800,50)(2200,40)};
\legend{PE/PP, PVC, Papier, Bois}
\fill[natrangreen!10] (axis cs:400,0) rectangle (axis cs:700,100);
\fill[natranorange!10] (axis cs:700,0) rectangle (axis cs:1000,100);
\fill[natranblue!10] (axis cs:1000,0) rectangle (axis cs:2500,100);
\node[font=\tiny] at (axis cs:550,92) {RGB};
\node[font=\tiny] at (axis cs:850,92) {NIR};
\node[font=\tiny] at (axis cs:1750,92) {SWIR};
\end{axis}
\end{tikzpicture}
\caption{Signatures spectrales des principaux matériaux CSR}
\end{figure}

\subsection{Modèle d'Intelligence Artificielle : YOLOv8}

\subsubsection{Choix du Modèle}

Nous avons choisi \textbf{YOLOv8} pour ses performances en détection d'objets en temps réel.

\begin{table}[H]
\centering
\begin{tabular}{|l|c|c|c|c|}
\hline
\textbf{Modèle} & \textbf{mAP@50} & \textbf{Vitesse (ms)} & \textbf{Paramètres} & \textbf{Adapté CSR} \\
\hline
YOLOv5s & 56.8\% & 6.4 & 7.2M & \cmark \\
YOLOv7 & 51.4\% & 8.7 & 36.9M & \cmark \\
\textbf{YOLOv8m} & \textbf{61.2\%} & \textbf{5.3} & \textbf{25.9M} & \cmark\cmark \\
YOLOv8x & 63.7\% & 12.1 & 68.2M & \cmark \\
\hline
\end{tabular}
\caption{Comparaison des modèles YOLO}
\end{table}

\begin{figure}[H]
\centering
\begin{tikzpicture}[scale=0.6, box/.style={rectangle, draw=natranblue, thick, fill=natranblue!10, minimum width=1.2cm, minimum height=0.6cm, align=center, font=\tiny}]
    \node[box, fill=natrangreen!20] (input) at (0,0) {Image 640×640};
    \node[box] (conv1) at (2,0) {Conv 3×3};
    \node[box] (c2f1) at (4,0) {C2f Block};
    \node[box] (sppf) at (6,0) {SPPF};
    \node[box, fill=natranorange!20] (fpn) at (8,0) {FPN/PAN};
    \node[box, fill=natranred!20] (head) at (10,0) {Detect Head};
    \node[box, fill=natrangreen!20] (output) at (12,0) {Classes + BBox};
    \draw[->, thick] (input) -- (conv1);
    \draw[->, thick] (conv1) -- (c2f1);
    \draw[->, thick] (c2f1) -- (sppf);
    \draw[->, thick] (sppf) -- (fpn);
    \draw[->, thick] (fpn) -- (head);
    \draw[->, thick] (head) -- (output);
    \node[font=\tiny, natrangray] at (3,-1) {Backbone};
    \node[font=\tiny, natrangray] at (8,-1) {Neck};
    \node[font=\tiny, natrangray] at (10,-1) {Head};
\end{tikzpicture}
\caption{Architecture simplifiée de YOLOv8}
\end{figure}

\subsubsection{Performances Simulées}

\begin{figure}[H]
\centering
\begin{tikzpicture}
\begin{axis}[
    width=10cm, height=6cm, ybar, bar width=15pt,
    ylabel={Pourcentage (\%)}, ymin=0, ymax=100,
    symbolic x coords={Précision, Rappel, F1-Score, mAP@50},
    xtick=data, nodes near coords, nodes near coords align={vertical},
    title={\textbf{Performances YOLOv8 vs Tri Manuel}},
    legend style={at={(0.5,-0.2)}, anchor=north},
]
\addplot[fill=natranblue!70] coordinates {(Précision,97.2)(Rappel,95.6)(F1-Score,96.4)(mAP@50,94.8)};
\addplot[fill=natranorange!70] coordinates {(Précision,85)(Rappel,80)(F1-Score,82.4)(mAP@50,78)};
\legend{YOLOv8 (simulé), Tri manuel (réel)}
\end{axis}
\end{tikzpicture}
\caption{Comparaison des performances}
\end{figure}

\subsection{Système Robotique}

\begin{table}[H]
\centering
\begin{tabular}{|l|c|}
\hline
\textbf{Caractéristique} & \textbf{Valeur} \\
\hline
Nombre de bras & 4 (2 accept + 2 reject) \\
Temps de cycle & 200-300 ms \\
Portée & 0.5 m \\
Charge utile & 2 kg \\
Précision & ± 2 mm \\
Durée de vie & 10 millions de cycles \\
\hline
\end{tabular}
\caption{Spécifications des bras robotiques}
\end{table}

\subsection{Système de Contrôle Qualité}

Le système QC est une innovation majeure permettant de corriger les erreurs en temps réel.

\begin{figure}[H]
\centering
\begin{tikzpicture}[scale=0.7]
    \fill[natrangray!30] (-4,2) rectangle (4,2.4);
    \node[font=\tiny] at (0,2.6) {Convoyeur Principal};
    \draw[natranblue, thick, dashed] (-2.5,1.5) rectangle (2.5,2.8);
    \node[font=\tiny, natranblue] at (0,3) {Zone Tri IA};
    \fill[natrangreen!30] (-4,0.5) rectangle (-1.5,0.9);
    \node[font=\tiny] at (-2.75,1.1) {QC Conformes};
    \fill[natranred!30] (1.5,0.5) rectangle (4,0.9);
    \node[font=\tiny] at (2.75,1.1) {QC Non-Conformes};
    \draw[natranorange, very thick] (-2.75,0) -- (-2.75,0.5);
    \node[font=\tiny, natranorange] at (-2.75,-0.3) {Bras QC FP};
    \draw[natrangreen!70!black, very thick] (2.75,0) -- (2.75,0.5);
    \node[font=\tiny, natrangreen!70!black] at (2.75,-0.3) {Bras QC FN};
    \fill[natrangreen!50] (-4,-1.2) rectangle (-2.5,-0.8);
    \node[font=\tiny] at (-3.25,-1) {Réacteur};
    \fill[natranred!50] (2.5,-1.2) rectangle (4,-0.8);
    \node[font=\tiny] at (3.25,-1) {Incinérateur};
\end{tikzpicture}
\caption{Système de contrôle qualité NATRAN}
\end{figure}

\begin{tcolorbox}[colback=natranorange!10, colframe=natranorange, title=Faux Positifs (FP)]
Objets \textbf{non-conformes} classés comme conformes par erreur. Le bras QC orange les éjecte vers le bac de rejet.
\end{tcolorbox}

\begin{tcolorbox}[colback=natrangreen!10, colframe=natrangreen, title=Faux Négatifs (FN)]
Objets \textbf{conformes} classés comme non-conformes par erreur. Le bras QC vert les récupère vers le réacteur.
\end{tcolorbox}

\section{Avantages de Notre Solution}

\subsection{Avantages Techniques}

\begin{table}[H]
\centering
\begin{tabular}{|l|p{5cm}|c|}
\hline
\textbf{Avantage} & \textbf{Description} & \textbf{Impact} \\
\hline
Précision élevée & 97.2\% vs 85\% manuel & +14\% \\
Haute cadence & Jusqu'à 100 obj/min & ×3 \\
Fonctionnement 24/7 & Pas de fatigue ni pause & +40\% productivité \\
Détection PVC & Imagerie SWIR spécifique & Chlore < 0.5\% \\
Contrôle qualité & Correction erreurs temps réel & -80\% erreurs \\
Traçabilité & Logs de chaque objet trié & 100\% traçable \\
\hline
\end{tabular}
\caption{Avantages techniques de NATRAN}
\end{table}

\subsection{Avantages Économiques}

\begin{figure}[H]
\centering
\begin{tikzpicture}
\begin{axis}[
    width=11cm, height=6cm,
    xlabel={Années}, ylabel={Coût cumulé (k€)},
    xmin=0, xmax=5, ymin=0, ymax=1200,
    legend style={at={(0.02,0.98)}, anchor=north west},
    grid=major, title={\textbf{Analyse TCO sur 5 ans}},
]
\addplot[natranred, thick, mark=square] coordinates {(0,0)(1,200)(2,400)(3,600)(4,800)(5,1000)};
\addplot[natranblue, thick, mark=*] coordinates {(0,250)(1,320)(2,390)(3,460)(4,530)(5,600)};
\legend{Tri manuel, NATRAN}
\draw[natrangreen, thick, dashed] (axis cs:1.5,0) -- (axis cs:1.5,350);
\node[font=\tiny, natrangreen] at (axis cs:1.8,380) {ROI 18 mois};
\end{axis}
\end{tikzpicture}
\caption{Analyse du coût total de possession (TCO)}
\end{figure}

\begin{tcolorbox}[colback=natrangreen!10, colframe=natrangreen, title=Économies Réalisées]
\begin{itemize}
    \item \textbf{Réduction main d'œuvre} : -60\% des coûts salariaux
    \item \textbf{Qualité CSR} : +15\% de valeur marchande
    \item \textbf{Pénalités évitées} : -90\% de non-conformités
    \item \textbf{ROI} : 18 mois en moyenne
\end{itemize}
\end{tcolorbox}

\section{Limites et Défis}

\subsection{Limites Techniques}

\begin{table}[H]
\centering
\begin{tabular}{|l|p{6cm}|p{4cm}|}
\hline
\textbf{Limite} & \textbf{Description} & \textbf{Mitigation} \\
\hline
Matériaux noirs & Absorption spectrale limitant la détection & Caméra thermique en option \\
\hline
Objets superposés & Difficulté de séparation & Pré-tri mécanique \\
\hline
Conditions variables & Poussière, humidité & Maintenance préventive \\
\hline
Nouveaux matériaux & Nécessite réentraînement & Mise à jour modèle \\
\hline
\end{tabular}
\caption{Limites techniques et mitigations}
\end{table}

\subsection{Limites Économiques}

\begin{itemize}
    \item \textbf{Investissement initial} : 200-350 k€ selon configuration
    \item \textbf{Compétences requises} : Formation du personnel technique
    \item \textbf{Dépendance technologique} : Maintenance spécialisée
    \item \textbf{Évolutivité} : Coût des mises à jour matérielles
\end{itemize}

\subsection{Défis d'Implémentation}

\begin{figure}[H]
\centering
\begin{tikzpicture}
\begin{axis}[
    width=10cm, height=6cm, xbar, bar width=10pt,
    xlabel={Niveau de difficulté (1-10)},
    symbolic y coords={Formation, Intégration IT, Installation, Calibration, Maintenance},
    ytick=data, nodes near coords,
    title={\textbf{Défis d'implémentation}},
    xmin=0, xmax=10,
]
\addplot[fill=natranorange!70] coordinates {(7,Formation)(6,Intégration IT)(5,Installation)(8,Calibration)(4,Maintenance)};
\end{axis}
\end{tikzpicture}
\caption{Évaluation des défis d'implémentation}
\end{figure}

\section{Types de CSR Traités}

\begin{table}[H]
\centering
\begin{tabular}{|l|l|c|c|c|c|}
\hline
\textbf{Type} & \textbf{Nom complet} & \textbf{Décision} & \textbf{PCI} & \textbf{Cl (\%)} & \textbf{Danger} \\
\hline
PE/PP & Polyéthylène/Polypropylène & \textcolor{natrangreen}{Accept} & 43 & 0 & Non \\
Carton & Carton/Papier souillé & \textcolor{natrangreen}{Accept} & 16 & 0 & Non \\
Bois & Bois traité/Palettes & \textcolor{natrangreen}{Accept} & 17 & 0 & Non \\
Textile & Textile non-chloré & \textcolor{natrangreen}{Accept} & 20 & 0 & Non \\
\hline
PVC & Polychlorure de vinyle & \textcolor{natranred}{Reject} & 18 & 57 & \textbf{Oui} \\
Métal & Métaux ferreux/non-ferreux & \textcolor{natranred}{Reject} & 5 & 0 & \textbf{Oui} \\
Verre & Verre/Céramique & \textcolor{natranred}{Reject} & 0 & 0 & Non \\
Caoutchouc & Caoutchouc/Pneus & \textcolor{natranred}{Reject} & 32 & 15 & \textbf{Oui} \\
\hline
PET & Polyéthylène téréphtalate & \textcolor{natranorange}{Incertain} & 23 & 0 & Non \\
Multi-couche & Emballage complexe & \textcolor{natranorange}{Incertain} & 25 & 5 & \textbf{Oui} \\
\hline
\end{tabular}
\caption{Classification des types de CSR traités par NATRAN}
\end{table}

\begin{figure}[H]
\centering
\begin{tikzpicture}
\begin{axis}[
    width=11cm, height=6cm,
    xlabel={PCI (MJ/kg)}, ylabel={Chlore (\%)},
    xmin=0, xmax=50, ymin=0, ymax=60,
    grid=major, title={\textbf{Cartographie des matériaux CSR}},
    legend style={at={(0.98,0.98)}, anchor=north east, font=\tiny},
]
% Zone acceptable
\fill[natrangreen!20] (axis cs:15,0) rectangle (axis cs:50,1);
% Zone rejet
\fill[natranred!20] (axis cs:0,1) rectangle (axis cs:50,60);
\fill[natranred!20] (axis cs:0,0) rectangle (axis cs:15,60);

% Points
\addplot[only marks, mark=*, mark size=3pt, natrangreen] coordinates {(43,0)(16,0)(17,0)(20,0)};
\addplot[only marks, mark=square*, mark size=3pt, natranred] coordinates {(18,57)(5,0)(0,0)(32,15)};
\addplot[only marks, mark=triangle*, mark size=3pt, natranorange] coordinates {(23,0)(25,5)};

\legend{Conformes, Non-conformes, Incertains}

% Labels
\node[font=\tiny] at (axis cs:43,3) {PE/PP};
\node[font=\tiny] at (axis cs:18,54) {PVC};
\node[font=\tiny] at (axis cs:32,18) {Caoutchouc};
\end{axis}
\end{tikzpicture}
\caption{Cartographie PCI/Chlore des matériaux CSR}
\end{figure}
