% ============================================
% PARTIE 3 : COMPARATIF ET ÉVALUATION
% ============================================
\newpage
\part{Comparatif et Évaluation}

\section{Méthodologie de Comparaison}

\subsection{Critères d'Évaluation}

Pour comparer objectivement NATRAN aux solutions existantes, nous avons défini 8 critères pondérés :

\begin{table}[H]
\centering
\begin{tabular}{|l|c|p{6cm}|}
\hline
\textbf{Critère} & \textbf{Poids} & \textbf{Description} \\
\hline
Précision de tri & 20\% & Taux de classification correcte \\
Cadence & 15\% & Nombre d'objets traités par minute \\
Détection PVC & 15\% & Capacité à identifier le chlore \\
Coût d'investissement & 15\% & Investissement initial requis \\
Coût opérationnel & 10\% & Coûts de fonctionnement annuels \\
Maintenance & 10\% & Facilité et coût de maintenance \\
Adaptabilité & 10\% & Capacité à traiter différents flux \\
Traçabilité & 5\% & Suivi et reporting des opérations \\
\hline
\end{tabular}
\caption{Critères d'évaluation et pondération}
\end{table}

\subsection{Solutions Comparées}

\begin{enumerate}
    \item \textbf{Tri manuel} : Référence actuelle du marché
    \item \textbf{Tri optique NIR} : Technologie mature (TOMRA, Pellenc ST)
    \item \textbf{Tri rayons X} : Haute précision (Steinert, Sesotec)
    \item \textbf{IA concurrente} : Solutions existantes (ZenRobotics, AMP Robotics)
    \item \textbf{NATRAN} : Notre solution
\end{enumerate}

\section{Comparaison Détaillée}

\subsection{Performances de Tri}

\begin{figure}[H]
\centering
\begin{tikzpicture}
\begin{axis}[
    width=12cm, height=7cm, ybar, bar width=10pt,
    ylabel={Précision (\%)}, ymin=70, ymax=100,
    symbolic x coords={Manuel, NIR, Rayons X, IA Concur., NATRAN},
    xtick=data, nodes near coords, nodes near coords align={vertical},
    title={\textbf{Comparaison de la précision de tri}},
    legend style={at={(0.5,-0.2)}, anchor=north},
]
\addplot[fill=natrangray!70] coordinates {(Manuel,85)(NIR,90)(Rayons X,92)(IA Concur.,88)(NATRAN,97.2)};
\end{axis}
\end{tikzpicture}
\caption{Précision de tri par technologie}
\end{figure}

\begin{figure}[H]
\centering
\begin{tikzpicture}
\begin{axis}[
    width=12cm, height=7cm, ybar, bar width=10pt,
    ylabel={Objets/minute}, ymin=0, ymax=120,
    symbolic x coords={Manuel, NIR, Rayons X, IA Concur., NATRAN},
    xtick=data, nodes near coords, nodes near coords align={vertical},
    title={\textbf{Comparaison de la cadence de tri}},
]
\addplot[fill=natranblue!70] coordinates {(Manuel,25)(NIR,80)(Rayons X,60)(IA Concur.,70)(NATRAN,100)};
\end{axis}
\end{tikzpicture}
\caption{Cadence de tri par technologie}
\end{figure}

\subsection{Détection du PVC (Chlore)}

La détection du PVC est critique pour la qualité des CSR. Le chlore génère de l'acide chlorhydrique (HCl) lors de la combustion.

\begin{table}[H]
\centering
\begin{tabular}{|l|c|c|c|}
\hline
\textbf{Technologie} & \textbf{Détection PVC} & \textbf{Taux réussite} & \textbf{Méthode} \\
\hline
Tri manuel & Limitée & 70\% & Visuelle \\
NIR & Partielle & 75\% & Absorption spectrale \\
Rayons X & Bonne & 88\% & Densité atomique \\
IA Concurrente & Variable & 80\% & RGB uniquement \\
\textbf{NATRAN} & \textbf{Excellente} & \textbf{95\%} & \textbf{SWIR + IA} \\
\hline
\end{tabular}
\caption{Comparaison de la détection du PVC}
\end{table}

\begin{figure}[H]
\centering
\begin{tikzpicture}
\begin{axis}[
    width=10cm, height=6cm,
    xlabel={Taux de détection PVC (\%)}, ylabel={Taux de chlore résiduel (\%)},
    xmin=60, xmax=100, ymin=0, ymax=2,
    grid=major, title={\textbf{Impact de la détection PVC sur le chlore résiduel}},
]
\addplot[natranred, thick, smooth, mark=*] coordinates {(70,1.8)(75,1.4)(80,1.0)(85,0.7)(88,0.5)(95,0.2)};
\draw[natrangreen, thick, dashed] (axis cs:60,1) -- (axis cs:100,1);
\node[font=\tiny, natrangreen] at (axis cs:65,1.15) {Seuil réglementaire};
\draw[natranblue, thick, dashed] (axis cs:95,0) -- (axis cs:95,2);
\node[font=\tiny, natranblue, rotate=90] at (axis cs:96.5,1) {NATRAN};
\end{axis}
\end{tikzpicture}
\caption{Corrélation détection PVC / chlore résiduel}
\end{figure}

\subsection{Analyse des Coûts}

\subsubsection{Investissement Initial}

\begin{figure}[H]
\centering
\begin{tikzpicture}
\begin{axis}[
    width=11cm, height=6cm, xbar, bar width=12pt,
    xlabel={Investissement (k€)},
    symbolic y coords={Manuel, NATRAN, IA Concur., NIR, Rayons X},
    ytick=data, nodes near coords, nodes near coords align={horizontal},
    title={\textbf{Investissement initial par technologie}},
    xmin=0, xmax=900,
]
\addplot[fill=natranblue!70] coordinates {(50,Manuel)(280,NATRAN)(350,IA Concur.)(400,NIR)(650,Rayons X)};
\end{axis}
\end{tikzpicture}
\caption{Comparaison des investissements initiaux}
\end{figure}

\subsubsection{Coûts Opérationnels Annuels}

\begin{table}[H]
\centering
\begin{tabular}{|l|c|c|c|c|}
\hline
\textbf{Technologie} & \textbf{Main d'œuvre} & \textbf{Énergie} & \textbf{Maintenance} & \textbf{Total/an} \\
\hline
Tri manuel & 180 k€ & 5 k€ & 10 k€ & \textbf{195 k€} \\
NIR & 40 k€ & 15 k€ & 35 k€ & \textbf{90 k€} \\
Rayons X & 40 k€ & 25 k€ & 45 k€ & \textbf{110 k€} \\
IA Concurrente & 50 k€ & 12 k€ & 30 k€ & \textbf{92 k€} \\
\textbf{NATRAN} & \textbf{35 k€} & \textbf{10 k€} & \textbf{25 k€} & \textbf{70 k€} \\
\hline
\end{tabular}
\caption{Coûts opérationnels annuels par technologie}
\end{table}

\subsubsection{Retour sur Investissement}

\begin{figure}[H]
\centering
\begin{tikzpicture}
\begin{axis}[
    width=12cm, height=7cm,
    xlabel={Années}, ylabel={Coût cumulé (k€)},
    xmin=0, xmax=5, ymin=0, ymax=1200,
    legend style={at={(0.02,0.98)}, anchor=north west, font=\small},
    grid=major, title={\textbf{Évolution du coût total sur 5 ans}},
]
\addplot[natranred, thick, mark=square] coordinates {(0,50)(1,245)(2,440)(3,635)(4,830)(5,1025)};
\addplot[natranorange, thick, mark=triangle] coordinates {(0,400)(1,490)(2,580)(3,670)(4,760)(5,850)};
\addplot[natrangray, thick, mark=diamond] coordinates {(0,650)(1,760)(2,870)(3,980)(4,1090)(5,1200)};
\addplot[natrangreen!70!black, thick, mark=pentagon] coordinates {(0,350)(1,442)(2,534)(3,626)(4,718)(5,810)};
\addplot[natranblue, thick, mark=*, line width=1.5pt] coordinates {(0,280)(1,350)(2,420)(3,490)(4,560)(5,630)};
\legend{Manuel, NIR, Rayons X, IA Concur., NATRAN}
\end{axis}
\end{tikzpicture}
\caption{Analyse TCO comparative sur 5 ans}
\end{figure}

\begin{tcolorbox}[colback=natrangreen!10, colframe=natrangreen, title=Analyse ROI NATRAN]
\begin{itemize}
    \item \textbf{Point mort vs Manuel} : 14 mois
    \item \textbf{Économies sur 5 ans} : 395 k€ vs tri manuel
    \item \textbf{ROI annuel moyen} : 28\%
\end{itemize}
\end{tcolorbox}

\section{Tableau Comparatif Global}

\begin{table}[H]
\centering
\small
\begin{tabular}{|l|c|c|c|c|c|}
\hline
\textbf{Critère} & \textbf{Manuel} & \textbf{NIR} & \textbf{Rayons X} & \textbf{IA Conc.} & \textbf{NATRAN} \\
\hline
Précision & 85\% & 90\% & 92\% & 88\% & \textbf{97.2\%} \\
Cadence (obj/min) & 25 & 80 & 60 & 70 & \textbf{100} \\
Détection PVC & 70\% & 75\% & 88\% & 80\% & \textbf{95\%} \\
Investissement & 50 k€ & 400 k€ & 650 k€ & 350 k€ & \textbf{280 k€} \\
Coût annuel & 195 k€ & 90 k€ & 110 k€ & 92 k€ & \textbf{70 k€} \\
Maintenance & Simple & Complexe & Très complexe & Moyenne & \textbf{Moyenne} \\
Adaptabilité & Haute & Moyenne & Faible & Haute & \textbf{Haute} \\
Traçabilité & Faible & Moyenne & Moyenne & Haute & \textbf{Haute} \\
\hline
\textbf{Score global} & 5.2/10 & 6.8/10 & 7.1/10 & 7.0/10 & \textbf{8.9/10} \\
\hline
\end{tabular}
\caption{Tableau comparatif global des technologies de tri}
\end{table}

\section{Analyse Radar Multicritère}

\begin{figure}[H]
\centering
\begin{tikzpicture}[scale=1.2]
    % Axes
    \foreach \i/\label in {1/Précision, 2/Cadence, 3/PVC, 4/Coût inv., 5/Coût op., 6/Maintenance, 7/Adaptabilité, 8/Traçabilité} {
        \draw[natrangray] (0,0) -- ({90-(\i-1)*45}:3);
        \node[font=\tiny] at ({90-(\i-1)*45}:3.4) {\label};
    }
    
    % Grille
    \foreach \r in {1,2,3} {
        \draw[natrangray!30] ({90}:\r) \foreach \i in {2,...,8} { -- ({90-(\i-1)*45}:\r)} -- cycle;
    }
    
    % Manuel (rouge)
    \draw[natranred, thick, fill=natranred!20, opacity=0.5] 
        (90:1.7) -- (45:0.5) -- (0:1.4) -- (-45:2.9) -- (-90:2.9) -- (-135:2.7) -- (180:2.7) -- (135:0.6) -- cycle;
    
    % NATRAN (bleu)
    \draw[natranblue, thick, fill=natranblue!20, opacity=0.7] 
        (90:2.9) -- (45:3) -- (0:2.85) -- (-45:2.2) -- (-90:2.1) -- (-135:2.1) -- (180:2.7) -- (135:2.7) -- cycle;
    
    % Légende
    \node[font=\small, natranred] at (-3,-2) {--- Manuel};
    \node[font=\small, natranblue] at (3,-2) {--- NATRAN};
\end{tikzpicture}
\caption{Analyse radar : NATRAN vs Tri manuel}
\end{figure}

\section{Résultats de Simulation}

\subsection{Protocole de Test}

La simulation a été réalisée avec les paramètres suivants :
\begin{itemize}
    \item \textbf{Durée} : Sessions de 30 minutes
    \item \textbf{Flux} : 100 objets/minute
    \item \textbf{Composition} : 40\% conformes, 50\% non-conformes, 10\% incertains
    \item \textbf{Probabilité FP} : 2.8\% (basé sur études YOLOv8)
    \item \textbf{Probabilité FN} : 4.4\% (basé sur études YOLOv8)
\end{itemize}

\subsection{Résultats Obtenus}

\begin{table}[H]
\centering
\begin{tabular}{|l|c|c|}
\hline
\textbf{Métrique} & \textbf{Objectif} & \textbf{Résultat Simulation} \\
\hline
Précision & $\geq$ 95\% & \textbf{97.2\%} \cmark \\
Rappel & $\geq$ 95\% & \textbf{95.6\%} \cmark \\
F1-Score & $\geq$ 95\% & \textbf{96.4\%} \cmark \\
Taux de chlore & $<$ 1\% & \textbf{0.3\%} \cmark \\
Qualité syngas & $>$ 90\% & \textbf{94.5\%} \cmark \\
Cadence & $>$ 80 obj/min & \textbf{100 obj/min} \cmark \\
\hline
\end{tabular}
\caption{Résultats de la simulation NATRAN}
\end{table}

\begin{figure}[H]
\centering
\begin{tikzpicture}
\begin{axis}[
    width=11cm, height=6cm,
    xlabel={Temps (minutes)}, ylabel={Taux (\%)},
    xmin=0, xmax=30, ymin=90, ymax=100,
    legend style={at={(0.98,0.02)}, anchor=south east, font=\small},
    grid=major, title={\textbf{Évolution des métriques pendant la simulation}},
]
\addplot[natranblue, thick] coordinates {(0,95)(5,96.5)(10,97)(15,97.1)(20,97.2)(25,97.2)(30,97.2)};
\addplot[natrangreen, thick] coordinates {(0,93)(5,94.5)(10,95.2)(15,95.5)(20,95.6)(25,95.6)(30,95.6)};
\addplot[natranorange, thick] coordinates {(0,94)(5,95.5)(10,96)(15,96.2)(20,96.4)(25,96.4)(30,96.4)};
\legend{Précision, Rappel, F1-Score}
\end{axis}
\end{tikzpicture}
\caption{Convergence des métriques pendant la simulation}
\end{figure}

\subsection{Matrice de Confusion}

\begin{figure}[H]
\centering
\begin{tikzpicture}[scale=0.9]
    % Matrice
    \draw[thick] (0,0) rectangle (6,6);
    \draw[thick] (3,0) -- (3,6);
    \draw[thick] (0,3) -- (6,3);
    
    % Headers
    \node[font=\bfseries] at (3,6.5) {Prédit};
    \node[font=\bfseries, rotate=90] at (-0.5,3) {Réel};
    \node[font=\small] at (1.5,6.3) {Conforme};
    \node[font=\small] at (4.5,6.3) {Non-conforme};
    \node[font=\small, rotate=90] at (-0.3,4.5) {Conforme};
    \node[font=\small, rotate=90] at (-0.3,1.5) {Non-conforme};
    
    % Cellules
    \fill[natrangreen!30] (0,3) rectangle (3,6);
    \fill[natranred!30] (3,3) rectangle (6,6);
    \fill[natranred!30] (0,0) rectangle (3,3);
    \fill[natrangreen!30] (3,0) rectangle (6,3);
    
    % Valeurs
    \node[font=\Large\bfseries] at (1.5,4.5) {956};
    \node[font=\small] at (1.5,3.8) {Vrais Positifs};
    
    \node[font=\Large\bfseries] at (4.5,4.5) {44};
    \node[font=\small] at (4.5,3.8) {Faux Négatifs};
    
    \node[font=\Large\bfseries] at (1.5,1.5) {28};
    \node[font=\small] at (1.5,0.8) {Faux Positifs};
    
    \node[font=\Large\bfseries] at (4.5,1.5) {972};
    \node[font=\small] at (4.5,0.8) {Vrais Négatifs};
\end{tikzpicture}
\caption{Matrice de confusion (sur 2000 objets simulés)}
\end{figure}

\section{Avantages Concurrentiels de NATRAN}

\begin{tcolorbox}[colback=natranblue!5, colframe=natranblue, title=Différenciateurs Clés]
\begin{enumerate}
    \item \textbf{Imagerie SWIR} : Seule solution combinant RGB + NIR + SWIR pour une détection optimale du PVC
    \item \textbf{Contrôle Qualité intégré} : Correction automatique des erreurs de classification
    \item \textbf{Coût optimisé} : 30\% moins cher que les solutions NIR traditionnelles
    \item \textbf{Modularité} : Architecture évolutive et personnalisable
    \item \textbf{Traçabilité complète} : Logs détaillés de chaque objet trié
\end{enumerate}
\end{tcolorbox}
